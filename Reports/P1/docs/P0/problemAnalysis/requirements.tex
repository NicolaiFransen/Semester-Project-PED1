\section{System requirements}

For the design and test of the MIC it is of great importance to have the requirements of the system defined. The input requirements of the MIC will be based on the specifications of the PV panel \textit{STP300S-20/Wfb} from Suntech Power. The specifications are shown in table \ref{el_charact_PV_panel}.

\begin{table}[htbp]
	\centering
	\begin{tabular}{ |l|c| } 
		\hline
		Maximum power ($P_{max}$) & 300 [W]  \\ \hline
		Optimum Operating Voltage ($V_{mpp}$) & 32.6 [V]  \\ \hline
		Optimum Operating Current ($I_{mpp}$) & 9.21 [A]  \\ \hline
		Open Circuit Voltage ($V_{oc}$) &  39.9 [V]\\ \hline
		Short Circuit Current ($I_{sc}$) & 9.65 [A]  \\ \hline
		Module Efficiency ($\eta$) & 18.3 \%  \\ \hline
	\end{tabular}
	\caption{Electrical characteristics \textit{STP300S-20/Wfb} \cite{PV_panel}.}
	\label{el_charact_PV_panel}
\end{table}

The values from the previous table will be the input for the DC-DC converter. The specifications of the load of the MIC will be based on the commercial inverter \textit{"Power-one STGU-105"}\cite{power_one_inverter} in order to have the output voltage defined. From the inverter's datasheet it is found that the nominal voltage in the DC-link is 360 V, with a maximum input power of 5500 W. \newline
The development of this project will be based on these requirements because they are based on real commercial products that the user can purchase. \todo{"Is based" 3 times in one paragraph.}


Table \ref{MIC_req} shows the requirements of the MIC, extracted from the specifications of the PV panel and the inverter. It defines both the requirements regarding input, output and of the length of PV panel strings. 
\begin{table}[htbp]
	\centering
	\begin{tabular}{|p{6cm}|>{\centering}p{8cm}|}
		\hline
		\rowcolor{lightgray}\multicolumn{2}{|l|}{ \textbf{Input}} \\ \hline
		Maximum input power ($P_{max}$) & 300 [W]  \tabularnewline \hline
		Maximum input Voltage ($V_{oc}$) & 40 [V]  \tabularnewline \hline
		Maximum input current ($I_{sc}$) & 10 [A]  \tabularnewline \hline
		Minimum efficiency ($\eta_{min}$) & 98 \%  \tabularnewline \hline
		
		\rowcolor{lightgray}\multicolumn{2}{|l|}{\textbf{Output}} \tabularnewline \hline
		Maximum output voltage ($V_{out}$) & 90 [V] \tabularnewline \hline
		Maximum output current ($I_{out}$) & 15 [A] \tabularnewline \hline
		
		\rowcolor{lightgray}\multicolumn{2}{|l|}{\textbf{Control}} \tabularnewline \hline
		Gain margin ($GM$) &  To be defined \tabularnewline \hline
		Phase margin ($PM$) & To be defined \tabularnewline \hline
		Rise time ($t_r$) & To be defined \tabularnewline \hline
		Overshoot ($OS$) & To be defined \tabularnewline \hline
		
		\rowcolor{lightgray}\multicolumn{2}{|l|}{\textbf{PV system specification}} \tabularnewline \hline
		Minimum string length & 4 \tabularnewline \hline
		Maximum string length & 15 \tabularnewline \hline
		
		\rowcolor{lightgray}\multicolumn{2}{|l|}{\textbf{Others}} \tabularnewline \hline
		Maximum dimensions & To be defined \tabularnewline \hline
		Operating Temperature & -40 to 85 [$\decC$] \tabularnewline \hline
	\end{tabular}
	\caption{MIC requirements.}
	\label{MIC_req}
\end{table}


%\textbf{Question supervisors:} According to these requirements could we estimate the number of panels to be used? We calculated the necessary duty cycles when working with n panels, and the result can be found in the table below. However we know that when working with duty cycle around 0.75 the currents get quite high, and that depends on the inductance and the switching frequency. Then should we wait until deciding this parameters for deciding the minimum number of panels or should we arbitrarily decide a minimum number of panels and then calculate switching frequency and inductance?

%\begin{table}[H]
%	\centering
%	\begin{tabular}{ |l|c|c|} 
%		\hline
%		Number of PV panels & Buck-boost Duty Cycle & Boost Duty Cycle  \\ \hline
%		1 & 0.91 & 0.91  \\ \hline
%		3 & 0.78 & 0.73 \\ \hline
%		5 &  0.69 & 0.54\\ \hline
%		10 & 0.52 & 0.1  \\ \hline
%	\end{tabular}
%\end{table}