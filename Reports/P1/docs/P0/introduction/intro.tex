To this date, sustainable energy sources have become an area of worldwide focus in an attempt to reduce the environmental impact due to emissions of $CO_{2}$ and other greenhouse gasses. The development of competitive systems to exploit renewable energy sources is the best alternative to reduce the use of fossil fuels for the production of electricity. Over the last years, there has been a considerable increase in electricity production from renewable energy sources being the fastest growing sectors wind and solar energy. In 2017, photovoltaic generation was the renewable energy source which experienced the highest increase in newly installed capacity. The total installed capacity reached approximately 402 GW\cite{global}. %[http://www.ren21.net/wp-content/uploads/2018/06/17-8652_GSR2018_FullReport_web_-1.pdf] 

Photovoltaic (PV) is referred to the production of electricity in the form of direct current (DC) directly from sunlight shining on solar cells. Solar cells are semiconductor devices which typically can produce around 0.5 V DC so they are series connected to form a PV panel which can also be connected to other PV panels resulting in a PV array \cite{handbook}. This way, according to the system's requirements, the PV panels can be interconnected in series or parallel in order to get at the output a higher voltage or current, respectively. Connecting PV panels either in series or parallel will result in an increase of the system's overall electricity production. %[http://www.sabz-energy.com/solar%20electricity%20handbook%202017.pdf] 

Nevertheless, it is essential to keep into consideration the mismatches that may appear on the power generated by the different PV panels. This will result in losses in the PV system and thus in a lower efficiency. Mismatches may lead to uneven power generation. These can be caused to partial shading, manufacturing tolerances, defects in the PV modules due to weather conditions and ageing among others. Even a small mismatch in one of the PV modules can result in a very high reduction of the power production from the entire PV array \cite{MPPmismatch}. Mismatch losses in a PV system can be reduced by forcing every PV module to work at its Maximum Power Point (MPP) by using a technique known as Maximum Power Point Tracking (MPPT). This can be reached by using electronic devices called Module Integrated Converters (MICs). MICs consist on DC-AC micro inverters or DC-DC converters that incorporate a MPPT controller unit to ensure that the output power of the MIC is the one corresponding to the MPP of the PV module \cite{MPPmismatch}.%[https://www.researchgate.net/publication/43248773_Study_on_MPP_mismatch_losses_in_photovoltaic_applications] 