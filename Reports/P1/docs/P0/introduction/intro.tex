To this date, sustainable energy sources have become an area of worldwide focus in an attempt to reduce the environmental impact due to emissions of $CO_{2}$ and other greenhouse gasses. The development of competitive systems to exploit renewable energy sources is the best alternative to reduce the use of fossil fuels for the production of electricity. Over the last years, there has been a considerable increase in electricity production from renewable energy sources being the fastest growing sectors wind and solar energy. In 2017, photovoltaic generation was the renewable energy source which experienced the highest increase in newly installed capacity. The total installed capacity reached approximately 402 GW\cite{global}. 

Photovoltaic (PV) is referred to the production of electricity in the form of direct current (DC) directly from sunlight shining on solar cells. Solar cells are semiconductor devices which typically can produce around 0.5 V DC so they are connected to form a PV panel. These panels can also be connected to other PV panels resulting in a PV array \cite{handbook}. This way, according to the system's requirements, the PV panels can be interconnected in series or parallel in order to get at the output a higher voltage or current, respectively.

Nevertheless, it is important to keep into consideration the mismatches that may appear on the power generated by the different PV panels.  Mismatches can be caused by partial shading, manufacturing tolerances, defects in the PV modules due to weather conditions and ageing among others. Even a small mismatch can result in a high reduction of the power production from the entire PV array \cite{MPPmismatch}. This results in losses in the PV system which can be reduced by forcing every PV module to work at its Maximum Power Point (MPP) by using a Maximum Power Point Tracking (MPPT) unit. This can be reached by using electronic devices called Module Integrated Converters (MICs). MICs consist on DC-AC micro inverters or DC-DC converters that incorporate a MPPT controller. The MPPT is used to ensure that the power generated is the one corresponding to the MPP of the panel \cite{MPPmismatch}.