\section{Selection of topology}\label{selection_of_topology}
The selection of converter topology will be made based on the research made earlier in this chapter. The converter should be able to allow both a higher and a lower output than the input. This requirement will limit the buck and boost converters, which converts either up or down. This means that before the requirement is met, both a buck and boost converter must be a part of the implementation. This is not desirable, because it will introduce unnecessary work.  

The next requirement states that the converter should have as high an efficiency as possible. The flyback converter will have a lower efficiency than the buck-boost, because of the transformer. This will introduce a loss in the extra inductor winding, and a larger loss in the FET because of the turns ratio in the transformer. Using a 4 transistor buck-boost converter, instead of a 2 transistor, it is possible to further optimizing the power loss because of the use of FET's instead of diodes. 

The 4 transistor buck-boost converter does also have the advantage of being bidirectional. This means that it's possible to either extract current from the PV-module to the inverter at the output, or to inject current from the inverter to the PV-module. Due to that the PV-modules acts like LEDs, they will radiate an infrared light if current is injected. If the PV-modules are damaged in some way, i.e. having cracks, the radiation will be affected. This means that it is possible to discover faulty modules before efficiency drops. This will increase the overall efficiency of the system and ease the maintenance sequence significantly. 

The Bidirectional Non-Inverting Buck-Boost converter is chosen because of these arguments. However the bidirectional functionality will not be addressed in this project, but could be a part of further development of the converter. 






 