\section{Perturb and Observe implementation}\label{MPPTImplementation}

The basic operation of the P\&O algorithm consists of perturbing the operating voltage of the PV module by means of the variation of the duty cycle of the DC-DC converter. After each perturbation the power generated by the PV module is measured (observed) and stored in order to compare it with the previous value of the power. Based on the result of the comparison the MPP can be tracked by deciding if in the next perturbation the panel's voltage should be increased or decreased \todo{This first paragraph should be in the intro of the chapter. Make a merge to what Thassilo has written. Stef}.

There are different techniques for implement the P\&O algorithm according to the value of the variable controlled by the MPPT and also depending if the perturb value is fixed or variable \cite{implementationPandO}. The conventional P\&O algorithm uses a fixed perturb to generate a voltage or current reference signal for the outer control loop.\todo{check the current reference signal TL} The outer loop is used to control the switching of the DC-DC converter. Another way of implementing the conventional P\&O algorithm is using an adaptive perturb by setting the initial perturbation to 10\% of the open-circuit voltage ($V_{oc}$). After each iteration its value is decreased by 50\% until it reaches 0.5\%
of $V_{oc}$ \cite{implementationPandO}. A different technique consists on using the duty cycle of the converter as the variable controlled by the MPPT block and, therefore, avoiding the outer control loop.\todo{remove therefore or delete the comma} As with the conventional P\&O method this technique can also be implemented using fix or variable perturb step \cite{implementationPandO}. 

The P\&O algorithm that will be implemented in this project is the MPPT controlling directly the duty cycle and using a variable perturb step. It was decided to directly control the duty cycle to simplify the control system as it is not necessary to implement the outer controller.\todo{in this sentence are 3 times control remove one or two of them} On the other hand, an adaptive perturb is selected instead of a fixed one. The reason is that the MPPT takes longer time to reach the MPP, if a small perturb step is implemented. However, using a large perturb step the tracking would be faster but the oscillations around the MPP would be higher \cite{implementationPandO}. For this reason, it was decided to start with a perturbation step of 10\% of the $V_{oc}$ until the system reaches a certain value close to the MPP. At this point the perturbation step is iteratively reduced to one third of its previous value in order to reach accurately the MPP with lower oscillations \todo{Is it enough explanation for the variable step?? Stef}. Figure \ref{BD_POalgorithm} shows the implementation of the system in \textit{PLECS} including the MPPT controller unit which operates at a frequency of 100 Hz. 

\begin{figure}[H]
	\begin{center}
		\includegraphics[width=\textwidth]{../Pictures/BD_implementation_POalgorithm}
		\caption{Block diagram of the system including the MPPT.}
		\label{BD_POalgorithm}
	\end{center}	
\end{figure}

From the previous figure it is observed that the output of the MPPT block are the corresponding duty cycles for buck and boost mode. The perturb step is a control variable called \textit{valg} which corresponds to the transfer function of the power converter. 
The transfer function when it is operating in buck and in boost mode is shown in equations \ref{tfbuck} and \ref{tfboost}, respectively. Plotting this transfer functions and mapping them as shown in figure \ref{fig:mappingtf} it is possible to obtain the corresponding duty cycle for the buck or the boost mode. If the control variable from the MPPT is lower than 0.5  means that the output voltage is lower than the input voltage and thus the converter will work as a buck converter with duty cycle $D_{buck}=2\cdot valg$. On the other hand, if the control variable is $valg \geq 0.5$ means that the output voltage is higher or equal than the input voltage and, therefore, the converter will operate in boost mode with duty cycle $D_{boost}=2\cdot valg - 1$.\todo{Even though it was a big issue of the design of the program i don't think it is important to specify how the decision is taken. Maybe better to say "With the variables blabla and blabla the decision is taken wether to work on buck or boost mode" or something like that, otherwise it is not going to be understood. AT. I think if we don't explain this then the flow chart will not be clear enough. Stef} These duty cycles are used to generate the corresponding PWM signals according to the converter's mode of operation at each time. 

\vspace{1cm}
\begin{minipage}{0.3\linewidth}
	\begin{equation}	\label{tfbuck}
	\frac{V_o}{V_i} = D
	\end{equation}

\end{minipage}%
\begin{minipage}{0.5\linewidth}	
	\begin{equation}	\label{tfboost}
	\frac{V_o}{V_i}= \frac{1}{1-D}
	\end{equation}

\end{minipage}

\begin{figure}[H]
	\begin{center}
		\includegraphics[width=1\textwidth]{../Pictures/decision_mode_operation}
		\caption{Mapping to decide the mode of operation.}
		\label{fig:mappingtf} 
	\end{center}	
\end{figure}


The corresponding flow chart used for the implementation of the Perturb \& Observe algorithm is shown in figure \ref{fcfinal}. From the flow chart it can be observed that the MPPT is enabled when the panel's voltage has reached the value of the open-circuit voltage. This means that the MPPT detects when the different between the measured voltage and the previous voltage is less than 0.1 to start evaluating the voltage and power.\todo{we need the explanination for the relationship between d and voltage and open loop calculation TL }
It is important to notice from figure \ref{BD_POalgorithm} that the current measurement is carried out in the inductor instead of in the PV panel. This is done for possible future implementation of an outer control loop. For this reason, it is necessary to transform the measured current in order to get the corresponding measurement for the PV module's current. This current transformation is just necessary in the case of buck mode as explained in section \ref{current_sensor}. The MPPT evaluates if the converter is working in buck mode and if it is, it multiplies the measured current by the corresponding duty cycle $D_{buck}=2\cdot valg$.\todo{remove if it is, it multplies tL } In boost mode the average current through the inductor corresponds to the PV module's current. 

The operation of the algorithm, shown in figure \ref{fcfinal}, is an iterative process in which the values of the PV panel's voltage and power, before and after applying a voltage perturbation, are compared  in order to locate the point of operation.\todo{change the sentences } This way it is possible to decide if the panel's voltage has to be increased or decreased in order to achieve the MPP. Based on the PV characteristic curve shown in figure \ref{fig:mpp}, the following situations can occur:

\begin{enumerate}
	\item Increment of voltage and increment of power means that the point of operation is located to the left of the MPP. Therefore, the perturbance continues in the same direction (voltage is increased) with a fixed perturb step. 
	\item Increment of voltage and decrement of power means that the point of operation of the panel has gone from being located to the left of the MPP to the right of it. Therefore, the next perturbance is in the opposite direction (voltage is decreased) with a perturb step of one third of the previous step value.
	\item Decrement of voltage and increment of power means that the operation point is located to the right of the MPP. Therefore, the perturbance continues in the same direction (voltage is decreased) with a fixed perturb step. 
	\item Decrement of voltage and decrement of power means that the point of operation of the panel has changed from being located to the right of the MPP to the left of it. Therefore, the next perturbance is in the opposite direction (voltage is increased) with a perturb step os one third of the previous step value.
\end{enumerate}

After the process, once the MPP has been reached, the algorithm oscillates around this optimal point of operation. However in this case, as a variable perturb step is applied, the oscillations around the MPP will be much lower than using fixed perturb step.  


\begin{figure}[H]
	\begin{center}
		\includegraphics[width=0.88\textwidth]{../Pictures/P1/Flow_chart/2018_11_29_Flow_chart_MPPT_Buck_Bosst_converter.pdf}
		\caption{Flow chart for the Perturb \& Observe algorithm.}
		\label{fcfinal} 
	\end{center}	
\end{figure}

\iffalse
THINGS TO CHANGE IN THE ALGORITHM (code and flow chart):
\begin{itemize}
	\item Dmin=0.05 and Dmax=0.95 not necessary as D is not duty cycle! Check everywhere to change D for control variable. 
	\item Logic for the reset of deltaD when the system detects a change in irradiance. I think we shouldn't include this condition in the flow chart we can just show the results in graphs (in the next section) and we make the flow chart easier to read and just showing the main code. 
	\item Not necessary the start counter because with the counter for the open loop calculation the MPPT is enabled after the Voc has been reached. 
\end{itemize}
\fi


