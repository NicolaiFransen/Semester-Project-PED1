This chapter contains the control of the Buck-Boost converter, so that the PV-model works at the maximum power point during operation. To achieve this, a MPPT algorithm is implemented. The output signal from the MPPT controller can be the duty cycle or a reference voltage, depending which kind of controller that is implemented. If the output signal is the duty cycle, a control system is not necessary because the duty cycle signal can connect directly to the PWM. To get the duty cycle from the reference voltage firstly this value will subtracted with the voltage from the pv panel. After this voltage and current control will change the calculated voltage to the necessary duty cycle. The PWM indicates how long the MOSFET is switched on for a period of time, allowing the output voltage of the converter to be regulated.
The selection of the MPPT algorithm based on the knowledge from subchapter "Maximum Power Point Tracking techniques".
Constant voltage is not applied in the project work. One of the requirements of the work is that the algorithm can find the MPP despite changes in the environmental operation and this is not possible with an implementation of constant voltage. Since for the algorithm Incremental conductance the program needs more complex commands, perturb and observe was chosen because of the simpler implementation. For future purposes, the incremental conductance can be implemented to compare the efficiency and reliability of the two methods.
%At the beginning of the chapter different methods for MPPT are introduced. The subchapter "Selection of MPPT algorithm" describes the pros and cons of the methods and compares them. The result of this discussion should give us the best method for the converter. 
At the beginning of the chapter the implemented algorithm is followed is described with a flow chart. To validate the algorithm, a simulation of a Buck-Boost converter with the MPPT algorithm is created with the program \textit{PLECS}.
