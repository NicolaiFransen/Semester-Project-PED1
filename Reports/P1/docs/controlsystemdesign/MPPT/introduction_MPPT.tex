This chapter contains the control of the Buck-Boost converter, so that the PV-model works at the maximum power point during operation. To achieve this, a MPPT algorithm is implemented. The output signal from the MPPT controller can be the duty cycle or a reference voltage \todo{maybe add a brief why two possible options? AT}. The duty cycle can connect directly to the PWM \todo{can connect ¿? AT}. To get the duty cycle from the reference voltage firstly the reference voltage will subtracted with the voltage from the pv panel. After this the calculated voltage will change in the necessary duty cycle signal after the voltage and current controls \todo{a little messy sentence. AT}. The PWM indicates how long the MOSFET is switched on for a period of time, allowing the output voltage of the converter to be regulated.\newline
At the beginning of the chapter different methods for MPPT are introduced. The subchapter "Selection of MPPT algorithm" describes the pros and cons of the methods and compares them. The result of this discussion should give us the best method for the converter. Then the description of the implemented algorithm is followed by a flow chart. To validate the algorithm, a simulation of a Buck-Boost converter with the MPPT algorithm is created with the program Plecs.
%% I dont know how you write Buck-Boost_converter in LAtex