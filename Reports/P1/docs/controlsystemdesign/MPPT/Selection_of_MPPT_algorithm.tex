\section{Selection of MPPT algorithm}\label{MPPTselection}

%\begin{table}[H]
	%\centering 
	%\begin{tabular}{|p{2cm}|>{\centering}p{2cm}|>{\centering}p{2cm}|>{\centering}p{2cm}|>{\centering}p{2cm}|>{\centering}p{2cm}|}
	%	\hline
	%	\rowcolor{lightgray}{ \textbf{MPPT \newline Algorithm}} & 	
		%{ \textbf{measured parameters}} &
		%{ \textbf{Control complexibility}} &
		%{ \textbf{Oszillation around MPP}} &
     	%{ \textbf{Effiency}} &
	    %{ \textbf{Repeatibility changing the ambient}}
		%\tabularnewline  \hline
		%Constant voltage 	& Voltage 		& simple loop & no (it stops at the MPP) & 92\% & is no working at high differnece in temperatur and fpr partly shaded \tabularnewline \hline
		%Perturb and observe & Voltage (+ current) 	& PI contoller (medium) & yes & 95\% & it is going to reach the MPP \tabularnewline \hline
		%Incremental conductance & Voltage + current & PI controller (higher) & no (it stops at the MPP) & 95\% & it is going to reach the MPP 	\tabularnewline	
		
	%\end{tabular}
	%\caption{Summary the Pro and Contra from each %MPPT algroithm }
	%\label{tab:summaryMPPT}
%\end{table}

%\begin{table}[H]
%	\centering 
%	\begin{tabular}{|p{2cm}|>{\centering}p{2cm}|>{\centering}p{2cm}|>{\centering}p{2cm}|>{\centering}p{2cm}|>{\centering}p{2cm}|>{\centering}p{2cm}|>{\centering}p{2cm}|}
	%	\hline
	%	\rowcolor{lightgray}{ \textbf{MPPT \newline Algorithm}} & 	
	%	{ \textbf{Sensed Parameters}} &
	%	{ \textbf{Microcontroller Computation}} &
	%	{ \textbf{Periodic Tuning}} &
	%	{ \textbf{Realization}} &
	%	{ \textbf{Complexity}}&
	%	{ \textbf{Reliability}}		&
	%	{ \textbf{Overall Cost}}
	%	\tabularnewline  \hline
	%	Constant voltage 	& Voltage 		& absent/low & yes & Easy to implement & very simple but difficult to get a optimal $k_{1}$& Not accurate & Low \tabularnewline \hline
	%	Perturb and observe & Voltage 	& Varies & Low & No & Take time to implement & Medium & Not so much accurate & Low/ Medium \tabularnewline \hline
	%	Incremental conductance & Voltage + current & PI controller (higher) & no (it stops at the MPP) & 95\% & it is going to reach the MPP 	\tabularnewline	
		
	%\end{tabular}
%	\caption{Summary the Pro and Contra from each MPPT algroithm }
	%\label{tab:summaryMPPT}
%\end{table}



