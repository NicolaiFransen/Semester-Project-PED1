\section{Selection of MPPT algorithm}\label{MPPTselection}

\begin{table}[H]
	\centering 
	\begin{tabular}{|p{2cm}|>{\centering}p{2cm}|>{\centering}p{2cm}|>{\centering}p{2cm}|>{\centering}p{2cm}|>{\centering}p{2cm}|}
		\hline
		\rowcolor{lightgray}{ \textbf{MPPT \newline Algorithm}} & 	
		{ \textbf{measured parameters}} &
		{ \textbf{Control complexibility}} &
		{ \textbf{Oszillation around MPP}} &
     	{ \textbf{Effiency}} &
	    { \textbf{Repeatibility changing the ambient}}
		\tabularnewline  \hline
		Constant voltage 	& Voltage 		& simple loop & no (it stops at the MPP) & 92\% & is no working at high differnece in temperatur and fpr partly shaded \tabularnewline \hline
		Perturb and observe & Voltage (+ current) 	& PI contoller (medium) & yes & 95\% & it is going to reach the MPP \tabularnewline \hline
		Incremental conductance & Voltage + current & PI controller (higher) & no (it stops at the MPP) & 95\% & it is going to reach the MPP 	\tabularnewline	
		
	\end{tabular}
	\caption{Summary the Pro and Contra from each MPPT algroithm }
	\label{tab:summaryMPPT}
\end{table}
