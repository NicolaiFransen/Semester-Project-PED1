\section{Future work}
This section describes possible future work of the project. This includes parts that were beyond the scope of the project. Furthermore, it includes improvements that were discovered during both the development and testing period of the project.

%4- Recommendations for future work. Make general statements. Items: a)what study is needed, b)Methods to be used, c)what is needed for that study.

\subsection{Hardware improvements}
The driver circuits have been designed using isolated power supplies for the high side drivers. These are costly both in size and money. Then, it could be preferable to implement a bootstrap circuit for the drivers. An option for a driver including bootstrap is the UCC27211 \cite{boot_driver_datasheet}. This IC includes both a high and a low side driver, such that only two ICs would be necessary for the converter. 

Another improvement could be the optimization of the optocouplers by selecting a quad optocoupler, instead of four individual ICs. 

The output voltage sensor has not been used during the project development and thus it could also be removed.
 
%bootstrap, change driver to the A version due to voltage levels and then change the optocouupler to a quad version, control system powered from pv-panel, \dots

\subsection{Inductor design}
The inductor used for the converter is reused from an earlier project. Measurements show that its core is oversized regarding current since it does not saturate before $15A$ \ref{Coil problems}. To achieve an optimal inductor, it needs to be designed for the specific requirements for this converter. Both the core size and wire diameter depend on the wanted current rating of the converter\cite{underthehood}. Therefore it should be researched if a physically smaller coil is obtainable.

Another option for reducing the size of the inductor would be the increase of the switching frequency. This could be an important improvement regarding cost since the coil is one of the most expensive components of the design. This topic can also be addressed in future research.


\subsection{PCB improvements}
The PCB designed for this project is highly focused on easing the tests in the laboratory. For this, tests points have been located all around the PCB and LEDs were placed. Also, the width and clearance of the control traces have been oversized for easing the laboratory tests. The overlay was also not going to be visible so text had to be milled in the top layer. 

For a commercial version of the PCB, other considerations would be taken into account. By increasing the switching frequency of the converter, an improvement on passive components size could be achieved. Minimizing the packages of some ICs and getting rid of others, such as the output voltage sensor, would also reduce cost and size. Finally, by increasing the number of layers on the PCB, better performance and smaller size is achievable.

For these reasons, a commercial version of the product would be much smaller. Also, as explained in section \ref{PCB_Control}, the removal of the test points and the LEDs would entail a better isolation between the control and power sides.

%\subsection{Switching frequency}
%In accordance with the supervisors, the switching frequency for the converter was selected to be $50kHz$. No further considerations or calculations was made to justify this selection. To select a switching frequency with an optimal compromise between switching losses and component size, this will have to be researched. 

%\subsection{Component price, system size, optimization needed}
%Component prize and system size will be introduced during the other sections...

\subsection{Converter's efficiency}
The main purpose of the MIC is to maximize the power generation of a PV panel. To achieve a good performance of the whole system, the efficiency of the converter itself should be maximized. During the tests, the converter's efficiency was measured to be higher than 93\%. Other papers show that higher efficiency should be achievable. \cite{underthehood}  \cite{efficient_buckboost}


%\subsection{MPPT technique}
%Even though the P\&O algorithm was implemented, initial research showed that the incremental conductance method could have been a better option. With the Implementation of an incremental conductance algorithm the response is more smooth and the tracking of the operating point is more accurate comparing to the classic P\&O. Further research and simulations will have to be made for comparing the two methods. \cite{AdvantageIncC}

\subsection{V/I controller}
The developed MPPT algorithm controls the system's generated power. However, the transients are not considered. In order to properly govern transients a controller should be added. In that scenario, the output of the MPPT algorithm will be the desired PV voltage. This input voltage reference will be fed to the controller, which will output a duty cycle. The controller has a higher execution frequency than the MPPT. Then, it is able to control the system's transients as desired, which might be chosen by the designer by tuning appropriately the controller's parameters. The use of the controller will reduce the system's oscillations lowering wrong MPPT inferences. 
