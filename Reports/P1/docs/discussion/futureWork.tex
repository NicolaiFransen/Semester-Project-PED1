\section{Future work}
This section will describe the planned future work of the project. This includes parts that were prioritized low or Simplified to achieve a working converter. Furthermore it includes improvements that was discovered during both the development and testing period of the project.

%4- Recommendations for future work. Make general statements. Items: a)what study is needed, b)Methods to be used, c)what is needed for that study.

\subsection{MPPT technique}
Even though the P\&O algorithm was implemented, initial research showed that the incremental conductance method could have been a better option.  Further research and simulations will have to be made for comparing the two methods. \todo{Write about what could be gained from changing MPPT, NHF.}

\subsection{V/I controller}


\subsection{Hardware improvements}
In the first iterations of the converter design, the driver circuits have been designed using isolated power supplies for the high side drivers. These are costly both in size and money wise. Because of that it's preferable to implement a bootstrap circuit for the drivers. An option for a driver including bootstrap could be UCC27211 \cite{boot_driver_datasheet}. This includes both a high and low side driver, such that only to IC's are necessary for the converter. The input threshold is below $2.8V$ for both sides, which means that a $5V$ optocoupler could be used for isolation. For cost and size optimization the optocouplers should be one quad optocoupler, instead of four singles. 
 
%bootstrap, change driver to the A version due to voltage levels and then change the optocouupler to a quad version, control system powered from pv-panel, \dots

\subsection{Coil design}
The coil used for the converter is reused from an earlier project. Measurements shows that it's oversized regarding current ratings. To achieve an optimal coil it should be designed for this specific converter. Both the core size and wire diameter depends on the wanted current rating of the converter \cite{underthehood}. Because of this it will be possible to lower both the cost and size of the coil. 

\subsection{Switching frequency limitations}

%\subsection{Component price, system size, optimization needed}
%Component prize and system size will be introduced during the other sections...

\subsection{Efficiency}
The main purpose of the MIC is to maximize the output efficiency of a PV-panel. To achieve that, the efficiency of the converter itself should be maximized. During the tests, the efficiency was measured to \textit{ADD SOME NICE DATA}\todo{Insert measured efficiency, NHF.}. Other papers shows that an efficiency of up to $95-96\%$ should be achievable\cite{underthehood}, \cite{efficient_buckboost}. 