\subsection{Inductor}
\label{Coil problems}
Since the scope of the project did not cover the design of the inductor, it was decided to use an existing one. A coil was then selected with an approximated inductance of $1mH$.
The PCB was then designed for this initial inductor. However, after some  tests under currents flow it was shown that the coil was not maintaining its inductance due to saturation of the core. Also, it was reaching temperatures of over $100\dec C$. The inductance was then tested under different currents and the behavior was verified as shown in figure \ref{Coil comparison}.

It was then decided to use another inductor which would satisfy the requirements in all cases. The same tests were performed for the new inductor and it was shown that it would fulfill the requirements. 
%Also, some thermal tests were performed on the new inductor. With a constant current of $10A$, showing a temperature of $60\dec C$ after 10 minutes.

\begin{figure}[H]
	\begin{center}
		\includegraphics[width=1\textwidth]{docs/discussion/CoilTests/CoilComparison50kHzV2.png}
		\caption{Inductor comparison under changes on current flow (50kHz).}
		\label{Coil comparison}
	\end{center}	
\end{figure}

However, the size and weight of the new inductor are much higher than those of the previous one. This means that the footprint designed is not valid for the new coil and it hangs loose outside the PCB. All test results in the report has been performed with the new inductor. \todo{Describing which inductor is used for tests}

