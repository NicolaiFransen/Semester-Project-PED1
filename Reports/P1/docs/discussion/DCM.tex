\subsection{Discontinuous conduction mode}
\label{DCM_Discussion}

\todo{DCM content is here, consider revising location. AT}

According to chapter 5 of "Fundamentals of Power Electronics" \cite{Erickson} "the discontinuous conduction mode (DCM) arises when the switching ripple in an inductor current is large enough to cause the polarity of the applied switch current to reverse". 

In the case of the bidirectional non-inverting buck-boost, the MOSFETs do not stop reverse current flow like diodes would do, this fact makes the system behave in unexpected ways since current is able to flow through the coil in the opposite direction. An example of this could be found when working in buck mode with a very low power generation at the PV. In this case, since the current is not blocked by a diode, it falls bellow zero, current would then start flowing out of the output capacitor and into the inductor, charging it in the opposite direction. When MOSFET 1 is ON, the current then flows through it and into the input capacitor and the PV. 

The boundaries for entering this DCM scenario are dependent on the PV power and on the voltage at the output. As the power generated by the PV decreases due to a change on the environmental conditions, the current also drops. Under these circumstances there is a limit for the inductor to work in continuous conduction mode (CCM). If this limit is exceeded, the inductor would undergo negative current.

However, the conditions for the inductor to enter DCM are very exceptional. In the figures \ref{DCM_3D_Buck} and \ref{DCM_3D_Boost}, the inductor distance to reach DCM under different output voltages and generated powers is shown. The power plotted has been calculated at a constant temperature of $15\dec C$ and at changing irradiance from $5W/m^2$ to $100W/m^2$. All the calculations have been performed assuming MPP.
The z-axis of the plot represents the distance from zero to the lower peak of the current ripple.

In the figures, it is seen how DCM is only reached under very limited conditions, with very low power generated in the PV. Due to this, when the behavior is present, the current is very low and it is unlikely that the PV would be damaged from this issue \todo{sure¿?}. Nevertheless, it is a topic of great consideration for further research.

\begin{figure}[H]
	\begin{center}
	\includegraphics[width=1\textwidth]{../Pictures/Buck_DistanceToDCM.png}
		\caption{Boundary conditions for DCM - Buck mode.}
		\label{DCM_3D_Buck}
	\end{center}	
\end{figure}



\begin{figure}[H]
	\begin{center}
		\includegraphics[width=1\textwidth]{../Pictures/Boost_DistanceToDCM.png}
		\caption{Boundary conditions for DCM - Boost mode.}
		\label{DCM_3D_Boost}
	\end{center}	
\end{figure}


