\pdfbookmark[0]{English title page}{label:titlepage_en}
\aautitlepage{%
  \englishprojectinfo{
    DC-DC Converter for PV Module Integration %title
  }{%
    Dynamics in Electrical Energy Engineering  %theme
  }{%
    Fall Semester 2018 %project period
  }{%
    INTRO-760 % project group
  }{%
    %list of group members
    Estefanía Ruiz \\ 
    Aitor Teran\\
    Nicolai Fransen \\ 
    Jesper Kloster\\
    Nicolás Murguizur Bustos\\
    Thassilo Lang
  }{%
    %list of supervisors
    Lajos Török\\
    Dezso Sera
  }{%
    1 % number of printed copies
  }{%
    \today % date of completion
  }%
}{%department and address
  \textbf{Department of Energy Technology}\\
  Aalborg University\\
  \href{http://www.aau.dk}{http://www.aau.dk}
}{% the abstract
  PV modules must work at their Maximum Power Point (MPP) to achieve the highest efficiency. Several environmental conditions like partial shading and temperature change can reduce the output power of the PV modules. The goal of this project is to design a Module Integrated Converter (MIC) for a PV module. It should be able to drive the PV at its MPP when the weather conditions change. A MIC consist of a DC-DC converter and a Maximum Power Point Tracking (MPPT) algorithm. It is decided to implement a non-inverting buck-boost converter with a Perturb and Observe (P\&O) MPPT algorithm. The objective of the project is to be able to drive a PV module at its MPP constantly regardless of the environmental conditions.
}