\pdfbookmark[0]{English title page}{label:titlepage_en}
\aautitlepage{%
  \englishprojectinfo{
    DC-DC Converter for PV Module Integration %title
  }{%
    Dynamics in Electrical Energy Engineering  %theme
  }{%
    Fall Semester 2018 %project period
  }{%
    INTRO-760 % project group
  }{%
    %list of group members
    Estefanía Ruiz \\ 
    Aitor Teran\\
    Nicolai Fransen \\ 
    Jesper Kloster\\
    Nicolás Murguizur Bustos\\
    Thassilo Lang
  }{%
    %list of supervisors
    Lajos Török\\
    Dezso Sera
  }{%
    1 % number of printed copies
  }{%
    \today % date of completion
  }%
}{%department and address
  \textbf{Department of Energy Technology}\\
  Aalborg University\\
  \href{http://www.aau.dk}{http://www.aau.dk}
}{% the abstract
  PV modules must work at their Maximum Power Point (MPP) to achieve the highest efficiency. Several environmental conditions, like partial shading and temperature increase, reduce the output power of the PV modules. In this project, a Module Integrated Converter (MIC) for a PV module was designed. It drives the PV panel at its MPP when the weather conditions change. A MIC consist of a DC-DC converter and a Maximum Power Point Tracking (MPPT) controller. It is decided to implement a bidirectional non-inverting buck-boost converter with a Perturb and Observe (P\&O) MPPT algorithm. A PCB was built for validating the system's design. 
  
  Keywords: MPPT, MIC, P\&O, bidirectional non-inverting buck-boost converter.
   
}