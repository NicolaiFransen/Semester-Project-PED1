\pdfbookmark[0]{English title page}{label:titlepage_en}
\aautitlepage{%
  \englishprojectinfo{
    DC-DC Converter for PV Module Integration %title
  }{%
    Dynamics in Electrical Energy Engineering  %theme
  }{%
    Fall Semester 2018 %project period
  }{%
    INTRO-760 % project group
  }{%
    %list of group members
    Estefanía Ruiz, 
    Aitor Teran\\
    Nicolai Fransen, 
    Jesper Kloster\\
    Nicolás Murguizur Bustos\\
    Thassilo Lang
  }{%
    %list of supervisors
    Lajos Török\\
    Dezso Sera
  }{%
    1 % number of printed copies
  }{%
    \today % date of completion
  }%
}{%department and address
  \textbf{Department of Energy Technology}\\
  Aalborg University\\
  \href{http://www.aau.dk}{http://www.aau.dk}
}{% the abstract
  PV modules must work at their Maximum Power Point (MPP) to achieve the highest efficiency. Several environmental condition like partial shading can reduce the output power of the PV modules. Hence the goal of this project work is to design a Module Integrated Converter (MIC) in a PV module. A MIC consist on a DC-DC converter and a Maximum Power Point Tracking (MPPT) controller. It is decided to implement a non-inverting buck-boost converter with a Perturb and Observe (P\&O) MPPT algorithm. The objective is that the PV module works at its MPP constantly regardless the environmental conditions.
}