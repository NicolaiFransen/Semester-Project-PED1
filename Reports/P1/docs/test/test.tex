%%% Chapter 7 - Test and validation of results %%%

\chapter{Test and validation of results} \label{ch:test_validation}

When the design stage is completed, it must be validated. The system design consisted of hardware and control design. The validation checks whether the system is compliant with the initial goals. The validation is performed in two steps: first the design is validated through software simulation. This step was explained in previous chapters. The second step consists on the validation over the physical implementation and will be addressed in this chapter.

The hardware design testing is performed following an incremental fashion. The components are individually soldered to the PCB and tested. The tests are short and with clear boundaries. This features ease the errors' finding and troubleshooting. This procedure validates the proper behaviour in the single component but also lets analyse the effect of additional components. The assembly and test procedure will be defined prior to the PCB assembly. The tests and its expected results are clear before starting. When the PCB is assembled and all the integration tests are passed, the complete converter is tested. This system test is performed by running the converter with a fixed duty cycle. This test allows the validation of the converter as a whole, including voltage and current ripples, thermal test, gain and losses.

The control design testing must demonstrate that the control signals are properly calculated in order to achieve maximum power generation. This signals are received by the previously tested hardware implementation. The testing consists on black box testing where only a few internal parameters are monitored through the controller's console. The 
system's stability and dynamic behaviour will analysed.

%% Section decribing the PCB building of the first iteration %%
\section{PCB building} \label{sec:pcb_building}
The building of the PCB in iteration one of the development has been done according to the test described in the following section. It has been divided into smaller tests, to validate that every part of the system works properly. The first part is the test of the control part, so the optocouplers, drivers and sensors. The last part is then the test of the power stage of the converter. It has been chosen only to solder the buck leg of the converter to validate the design, and discover faults which must be dealt with in iteration two. Before starting the test all vias, capacitors and test-points are soldered to assure connectivity.

\subsection{Power supplies} \label{sec:test_pwr_sup}
The purpose of the first test was to validate the different power supplies in the converter. For this test it's necessary to solder the in-TRACO and the $5V$ voltage regulator. The three LED's and the resistors should also be soldered. Apply $5V$ at connector J4 and $12V$ at connector J2. The test is conducted by following table \ref{tab:test_pwr_sup}

\begin{table}[H]
	\centering
	\begin{tabular}{|>{\centering}p{1cm}|p{7cm}|p{4cm}|>{\centering}p{2cm}}
		\hline
		\rowcolor{lightgray}\multicolumn{4}{|l|}{ \textbf{Test of power supplies}} \\ \hline
		\rowcolor{lightgray} \textbf{ID} & \textbf{Test} & \textbf{Test-points} & \textbf{Pass/Fail} \tabularnewline \hline
		1.1 & All LED's must be lit & DS1-2-4 & Pass  \tabularnewline \hline
		1.2 & Measure $5V$ at low-voltage & 5V-LV \& GND-LV & Pass \tabularnewline \hline
		1.3 & Measure $12V$ at high voltage, low-side & 12V \& GND-in & Pass  \tabularnewline \hline
		1.4 & Measure $12V$ at output of in-TRACO & 12V-in \& L-in & Pass  \tabularnewline \hline
		1.5 & Measure $5V$ at the output of the voltage regulator & 5V-HV \& GND-sen & Pass  \tabularnewline \hline
	\end{tabular}
	\caption{Test of power supplies.}
	\label{tab:test_pwr_sup}
\end{table}

\subsection{Optocouplers} \label{sec:test_opto}


\section{MPPT}

\subsection{RT-box}

\subsection{PV simulator}

\subsection{Load}

\subsection{Experimental results}