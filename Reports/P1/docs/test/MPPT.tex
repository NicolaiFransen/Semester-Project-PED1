\section{System test}
After the Assembling of the PCB and testing the components behave like the expectation, the whole converter will be tested about the system test. At first the converter is subjected the open loop test. The reason is to compare the results with the values in the ideal open loop simulation in section \ref{opsimresult}. The second test includes the thermal test. The converter is performed for a longer time. This will be done to identify which parts of the PCB will produce too much heat to damage other components. The last test is about checking the P\&O MPPT algorithm, if it behaves like same as in the simulation from section \ref{MPPTSimulation}.

\subsection{Open loop test}
For the open loop test the converter is tested one time in buck and boost mode. Both mode will be test at the maximum power point. So, the constant input voltage for both test is 36.9 V. The environmental condition in these cases is 1000 $W /m^2$ for the irradiance 1000 $W /m^2$ and for the temperature 25 $\decC$. For both test it will measure the current ripple and the output voltage to check if they have same value as calculate. Also it will validate the steady state of the experimental result with the values from the simulation.

For the buck mode the load at the output is 3 Ohm. With the fix duty cycle value, the output voltage should be the value in steady state $V_{out} = 24 V$. The figure shows the transition to reach the steady state. Here a sentence about the comparing with the simulation.  In the figure the output voltage at steady state is 21.1 V less than the expected value. The efficiency of the system is 87.5$\%$. The main losses are the dissipation from the component in heat. Also the load is a power resistor and it change the resistance if it is working. So we have not a constant load at the output as in the simulation. Another reason is the dead-time of the MOSFET and therefore the duty cycle is not exactly the same value as in the simulation. 

\begin{itemize}
	\item Graph current in the inductor and output voltage to validate figure 3.2. 
	\item Graph current in the inductor and input voltage to check the ripple. 
\end{itemize}

\subsection{Thermal test}

\subsection{MPPT}

\begin{itemize}	
	\item Set irradiance 1000 $W /m2$ and T= 25 $\decC$. This is: 
		\begin{itemize}
		\item Isc = 8.67 A (in the simulator in the lab set this current to the max which is 8.6 A), Voc=45 V, Impp=8.14 A and Vmpp=36.9 V.
		\end{itemize} 
	\item For load resistance R=3 $\Omega$ (buck) and R=27 $\Omega$ (boost), plot the next graphs in the oscilloscope:
		\begin{enumerate}
		\item Input voltage vs Input current 
		\item Input power. Pmpp=300.4W
		\item XY graph of Vin and Iin. I-V curve
		\item Input voltage vs Output voltage
		\item Duty cycles for buck and boost mode using the Analog output from RT-box. 
		\item Optional plot input power vs output power to show the efficiency of the converter.
		\end{enumerate}
	
	\item Set the PV simulator to detect a step change in irradiance keeping T=25 $\decC$. Save 2 lists of data, one for 1000 $W /m2$ of irradiance (same values as before) and the other for 800 $W /m2$. This is:
		\begin{itemize}
		\item Isc = 6.94 A, Voc= 44.52 V, Impp= 6.49 A and Vmpp= 36.9 V.
      	\end{itemize} 
      \item For load resistance R=3 $\Omega$ (buck) and R=27 $\Omega$ (boost), plot the next graphs in the oscilloscope:
      \begin{enumerate}
      	\item Input voltage vs Input current 
      	\item Input power. Pmpp = 300.4 to Pmpp = 240W
      	\item XY graph of Vin and Iin. I-V curve
      	\item Optional plot input power vs output power to show the efficiency of the converter.
      \end{enumerate}
  
  \item Set the PV simulator to detect a step change in temperature keeping irradiance to 1000 $W /m2$.
  Save 2 lists of data, one for T=25$\decC$ (same values as before) and the other for T=20$\decC$. This is:\todo{In simulations the step change in temperature is 25- 35$\decC$, however, for 35$\decC$ the short circuit current is 9.5 which is higher than the max. allowed for the PV simulator(8.6A) so a change to a higher temperature would not be possible in the lab.}
  \begin{itemize}
  	\item Isc = 8.2 A, Voc= 47.1 V \todo{We must check if the PV simulator can work with these values}, Impp= 7.75 A and Vmpp= 39.1 V.
  \end{itemize} 

  \item For load resistance R=3 $\Omega$ (buck) and R=27 $\Omega$ (boost), plot the next graphs in the oscilloscope:\todo{In case we can simulate this change, modify the graphs obtained in simulation for 25-35 $\decC$}
  \begin{enumerate}
  	\item Input voltage vs Input current 
  	\item Input power. Pmpp=300.4 to  Pmpp=303.2W
  	\item XY graph of Vin and Iin. I-V curve
  	\item Optional plot input power vs output power to show the efficiency of the converter.
  \end{enumerate}
  
  
\end{itemize}
	
\subsection{Experimental results}