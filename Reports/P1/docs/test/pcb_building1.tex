\section{PCB building} \label{sec:pcb_building}
The building of the PCB in iteration one of the development has been done according to the test described in the following section. It has been divided into smaller tests, to validate that every part of the system works properly. The first part is the test of the control part, so the optocouplers, drivers and sensors. The last part is then the test of the power stage of the converter. It has been chosen only to solder the buck leg of the converter to validate the design, and discover faults which must be dealt with in iteration two. Before starting the test all vias, capacitors and test-points are soldered to assure connectivity.

\subsection{Power supplies} \label{sec:test_pwr_sup}
The purpose of the first test was to validate the different power supplies in the converter. For this test it's necessary to solder the in-TRACO and the $5V$ voltage regulator. The three LED's and the resistors should also be soldered. Apply $5V$ at connector J4 and $12V$ at connector J2. The test is conducted by following table \ref{tab:test_pwr_sup}

\begin{table}[H]
	\centering
	\begin{tabular}{|>{\centering}p{1cm}|p{7cm}|p{4cm}|>{\centering}p{2cm}}
		\hline
		\rowcolor{lightgray}\multicolumn{4}{|l|}{ \textbf{Test of power supplies}} \\ \hline
		\rowcolor{lightgray} \textbf{ID} & \textbf{Test} & \textbf{Test-points} & \textbf{Pass/Fail} \tabularnewline \hline
		1.1 & All LED's must be lit & DS1-2-4 & Pass  \tabularnewline \hline
		1.2 & Measure $5V$ at low-voltage & 5V-LV \& GND-LV & Pass \tabularnewline \hline
		1.3 & Measure $12V$ at high voltage, low-side & 12V \& GND-in & Pass  \tabularnewline \hline
		1.4 & Measure $12V$ at output of in-TRACO & 12V-in \& L-in & Pass  \tabularnewline \hline
		1.5 & Measure $5V$ at the output of the voltage regulator & 5V-HV \& GND-sen & Pass  \tabularnewline \hline
	\end{tabular}
	\caption{Test of power supplies.}
	\label{tab:test_pwr_sup}
\end{table}

\subsection{Optocouplers} \label{sec:test_opto}
