\subsection{Switch sizing} \label{switch_sizing}

The system must regulate the power flow in order to maximize the power generation \todo{power flow? Better the system must regulate input voltage and current to... AT. Discuss}. In order to achieve this, the system includes switches that control the current flow \todo{switches themselves don't control the current flow, the whole topology does so. AT Discuss}. The switches consist on MOSFET devices. The switching frequency of the system is $50 $ kHz. Although the market has IGBT which can switch at $50$ kHz, MOSFET devices allow lower losses than IGBTs for system's current rating. \cite{mosfet_igbt_switching_loss} \cite{igbt_or_mosfet}


The maximum output voltage of the system is $90 $V, however the voltage rating of the transistors was set to $150 $V in order to increase the reliability \todo{to have a safety margin to avoid damage of the switching device. Reliability? Stf}. The peak current through the transistors happens when the buck mode is activated. The peak current is equal to $14$ A\todo{check with jesper if hhe did this calculation and then link, if not I calculated and have a photo of it}. In order to reduce the conduction losses and the heat sink size, a low on resistance is desired. This constraints were used when searching for the ideal component. The chosen device is the IPB200N15N3 \todo{include datasheet AT}. It exhibits the features seen in table \ref{mosfet_features}.


\begin{table}[htbp]
	\centering
	\begin{tabular}{|p{6cm}|>{\centering}p{8cm}|}
		\hline
		\rowcolor{lightgray}\multicolumn{2}{|l|}{ \textbf{Maximum ratings}} \\ \hline
		Continuous $I_{D}$ & 40 [A]  \tabularnewline \hline
		$V_{GS}$ & $\pm$ 20 [V]  \tabularnewline \hline
		Power dissipation & 150 [W]  \tabularnewline \hline
		$V_{DS}$ & 150 [V]  \tabularnewline \hline
		$R_{DSon} $ & 20 [m$\Omega$]  \tabularnewline \hline
		\rowcolor{lightgray}\multicolumn{2}{|l|}{ \textbf{Other values of interest}} \\ \hline
		Input capacitance & 1820 [pF]  \tabularnewline \hline
		Package & D2PAK  \tabularnewline \hline
		$V_{th} $ $(V_{GS} = 3 V)$ & 3 [V]  \tabularnewline \hline
		$V_{th} $ $(V_{GS} = 36 V)$ & 4.7 [V]  \tabularnewline \hline
		$R_{Gate} $ & 2.4 [$\Omega$]  \tabularnewline \hline
	
	\end{tabular}
	\caption{MOSFET figures of merit. T = 25 $\decC$. \cite{mosfet_datasheet}}
	\label{mosfet_features}
\end{table}
\todo{Power dissipation 150W? what does that mean? AT Power dissipation is a standard datasheet parameter that gives an insight on the maximum power that the element can withstand, PD is calculated as the maximum power dissipation for a device with an infinite heat sink at 25C ambient.(easy to google)}

\subsection{Heat sink sizing}

The procedure followed for validating the heat sink might be seen at figure \ref{heat_sink_validation_procedure}. If the total temperature increase is within switch's safe operating area, then the heat sink is providing enough heat dissipation.

\begin{figure}[htbp]
	\begin{center}
		\includegraphics[width=\textwidth]{../Pictures/P1/Component_sizing/heat_sink_validation_procedure.png}
		\caption{Heat sink validation procedure.}
		\label{heat_sink_validation_procedure}
	\end{center}	
\end{figure}

The power dissipated in the switches is equal to the sum of the conduction losses and the switching losses. The conduction loss might be calculated  as seen in equation \ref{conduction_losses_eq}.

\begin{equation} \label{conduction_losses_eq}
P_{cond} = i(t)^2 \cdot R_{DS}
\end{equation}

The switching losses depend upon the switching frequency and the transistor's manufacturing characteristics. In order to calculate the value, the MOSFET's SPICE model was obtained from the manufacturer's website. The next step was to perform the simulation of the system. The system was simulated in both Buck and Boost modes. Special attention was put into the dead-band between PWM signals of different switches, to avoid current shoot through. After simulating, the average power dissipation under steady state was calculated in both Buck and Boost modes. Within every mode, the simulation was performed under the most unfavorable conditions, this is: Buck's output is 24 V and Boost's output is 90 V.\todo{Is it possible that low voltage at input, meaning low power may result in more unfavorable conditions? I mean, maybe going from 5V -> 90V is worst than 32V to 90V even if the power is less... AT. PV input voltage is always around 30-40 no matter what irradiance. NM} The results can be seen in table \ref{mosfet_final_dissipation}, column 1.

$R_{DS}$ varies  is mainly dependent on the temperature. The models found neglect the temperature difference. Then, in order to get an approximated value considering temperature, the procedure will be to calculate the total losses at constant temperature using the SPICE model and then add the additional conduction losses due to the increase of the resistance, as expressed by equation \ref{total_losses}.


\todo{Since you have made the simulations it would be a good idea to add them as appendices and refer to them. AT. Might be a good idea. Lets discuss it (maximum number of pages).}

\begin{equation} \label{total_losses}
\overline{P} = \overline{P_{loss, T = K}} + \overline{i(t)^2 \cdot \Updelta R_{DS}}
\end{equation}

Now the junction temperature based on the power dissipation calculated using the SPICE model is calculated. The ambient temperature is set to 50 $\decC$, which is considered a realistic scenario. The thermal circuit can be seen in figure \ref{thermal_circuit}. The next step is to choose a commercial heat sink. The constraints are thermal resistance, size and price. TDEX6015/TH was found. Its features might be found in table \ref{heatsink_features}. The switches temperature will be analysed in order to validate the heat sink. The analysis considers all the transistors as a single power source.

\begin{figure}[H]
	\begin{center}
		\includegraphics[width=0.7\textwidth]{../Pictures/thermal_circuit.png}
		\caption{Thermal circuit used for sizing the heat sink}
		\label{thermal_circuit}
	\end{center}	
\end{figure}

\begin{equation} \label{switch_temperature}
T_{J} = T_{housing} + \overline{P_{loss, T = K}} \cdot  R_{thermal}
\end{equation}
\todo{Also include units in the lower equation.AT I think thats something that we should agree. NM}

If no heat sink were used, according to equation \ref{switch_temperature}, the junction temperature would become too high and the components would be damaged. See equation \ref{temperature_without_heatsink}. This is mainly explained due to the fact that the thermal resistance between junction and ambient of the transistor is as high as 75 $\decC / W$.

\begin{equation} \label{temperature_without_heatsink}
T_{J} = 50 \decC + 5.54 W \cdot 75 \frac{\decC}{W} = 465.5 \decC
\end{equation}


\begin{table}[htbp]
	\centering
	\begin{tabular}{|p{6cm}|>{\centering}p{8cm}|}
		\hline
		\rowcolor{lightgray}\multicolumn{2}{|l|}{ \textbf{Features}} \\ \hline
		Size & 60x60x16 [mm]  \tabularnewline \hline
		Thermal resistance & 2.06 [K/W]  \tabularnewline \hline
		
	\end{tabular}
	\caption{Heat sink figures of merit. \cite{heatsink_datasheet}}
	\label{heatsink_features}
\end{table}


\begin{equation} \label{switch_temperature_w_values}
T_{J} = 50 \decC + 5.54 W \cdot  2.06\frac{\decC}{W} = 61.41 \decC
\end{equation}\todo{missing units in the eq. Where do the 5.54 and 2.06 come from? Stef}

The Drain to Source resistance increase is calculated as explained in equation \ref{delta_resistance}. The resistance difference is relatively small.

\begin{equation} \label{delta_resistance}
\Updelta R_{DS} = |R_{DS, T = 20 \decC} - R_{DS, T = 61.41\decC}| = 4\; m \Omega
\end{equation}\todo{what are the values? you got them in simulation? Stef}

\begin{table}[]
	\centering
	\begin{tabular}{|l|l|l|l|}
		\hline
		\rowcolor[HTML]{C0C0C0} 
		\multicolumn{4}{|c|}{\cellcolor[HTML]{C0C0C0}\textbf{Switches power dissipation}}                                                   \\ \hline
		\rowcolor[HTML]{C0C0C0} 
		Switch         & $\overline{P_{loss, T = K}}$ {[}W{]} & $ \overline{i(t)^2 \cdot \Updelta R_{DS}}$ {[}W{]} & \textbf{Total {[}W{]}} \\ \hline
		\multicolumn{4}{|l|}{Buck mode}                                                                                                     \\ \hline
		M1             & 2.91                                 & 0.39                                               & \textbf{3.30}          \\ \hline
		M2             & 0.82                                 & 0.21                                               & \textbf{1.03}          \\ \hline
		M3             & 1.81                                 & 0.58                                               & \textbf{2.39}          \\ \hline
		M4             & 0                                    & 0                                                  & \textbf{0}             \\ \hline
		\textbf{Total} & 5.54                                 & 1.18                                               & \textbf{6.72}          \\ \hline
		\multicolumn{4}{|l|}{Boost mode}                                                                                                    \\ \hline
		M1             & 0.69                                 & 0.28                                               & \textbf{0.97}          \\ \hline
		M2             & 0                                    & 0                                                  & \textbf{0}             \\ \hline
		M3             & 0.48                                 & 0.12                                               & \textbf{0.6}           \\ \hline
		M4             & 3.31                                 & 0.18                                               & \textbf{3.49}          \\ \hline
		\textbf{Total} & 4.48                                 & 0.58                                               & \textbf{5.06}          \\ \hline
	\end{tabular}
\caption{Power dissipation analysis. Column 1, average power dissipation at constant 25 $\decC$ temperature. Column 2, extra power dissipation due to the increase of temperature.}
\label{mosfet_final_dissipation}
\end{table}


The full power dissipation values can be found on table \ref{mosfet_final_dissipation}. To achieve an exact result, the new total power should be used with the thermal circuit in order to calculate again $\Updelta R_{DS}'$, but this difference is small and thus neglected \todo{if it's neglected then it should not be calculated AT. lets talk about this}. Now that the power dissipation has been calculated, the junction temperature must be checked in order to confirm that the heat sink has been properly sized. Equation \ref{switch_temperature} is used, substitution of values leads to \ref{switch_temperature_w_values_2}. The difference is fairly small and the junction temperature remains within safe area. Then, TDEX6015/TH has been validated as a proper heat sink.

\begin{equation} \label{switch_temperature_w_values_2}
T_{J} = 50 \decC + 6.72 W \cdot  2.06 \frac{\decC}{W} = 63.84 \decC
\end{equation}

\todo{I dont understand the difference between equation 4.10 and 4.12. It seems that you have recalculated the losses but in the previous paragraph you state that the difference has been neglected. We should make it more clear. AT }

\todo{So when i went to the next chapter i saw that you included another table with new values for power dissipation under temperature increase. I think it is located in the wrong place but also it should have the same appearance than the previous one, maybe best if both have the data that the second one has. AT}
