\subsubsection{Input voltage sensor} \label{voltage_sensors}
The voltage sensor selected is the ACPL-C870 \cite{voltage_sensor}. This sensor includes an optical isolation amplifier, which makes it well suited for isolated voltage sensing. Some relevant electrical specifications have been included in table \ref{tab:voltage_sensor_features}.


\begin{table}[H]
	\centering
	\begin{tabular}{|p{6cm}|>{\centering}p{3.5cm}|>{\centering}p{3.5cm}|}
		\hline
		\rowcolor{lightgray}\multicolumn{3}{|l|}{ \textbf{Recommended ratings}} \\ \hline
		Supply voltages 	& $V_{DD1}, V_{DD2}$ & $5$ [V]  \tabularnewline \hline
		Input voltage range & $V_{in}$ 			 & $0-2$  [V]  \tabularnewline \hline
		
		\rowcolor{lightgray}\multicolumn{3}{|l|}{ \textbf{Other values of interest}} \\ \hline
		Voltage gain 		& $G$ 				 & 1 [V/V]  \tabularnewline \hline
		Output common-mode voltage & $V_{OCM}$ & 1.23 [V]  \tabularnewline \hline
		Gain tolerance & -- & $\pm 3$ [$\%$]  \tabularnewline \hline
		Bandwidth 		& $BW$ & $100$ [kHz]	\tabularnewline \hline
		Package & SSOP & [-] \tabularnewline \hline
		
	\end{tabular}
	\caption{Electrical specifications ACPL-C870 \cite{voltage_sensor}.}
	\label{tab:voltage_sensor_features}
\end{table}

\paragraph{The circuit}
The circuit regarding the input voltage measurement is shown at figure \ref{fig:input_voltage_sensor_circuit}. $V_{in}$ is the measured voltage from the PV module. The points $OA1_+$ and $OA1_-$ are connected to the non-inverting and inverting input of the amplifier respectively. $OA1_{out}$ is connected to the output of the amplifier. $C_{15}$ and $C_{16}$ are decoupling capacitors for the two supply voltages. The maximum input voltage of the voltage sensor is $5V$. Because of this a $4.7V$ zener diode has been added at the input, to protect it against over-voltage.

\begin{figure}[H]
	\begin{center}
		\includegraphics[width=0.8\linewidth]{../Pictures/P1/Sensors/input_voltage_sensor.PNG}
		\caption{Input voltage sensor.}
		\label{fig:input_voltage_sensor_circuit}
	\end{center}
\end{figure}

\iffalse
\begin{figure}[H]
	\begin{center}
		\includegraphics[width=0.7\linewidth]{../Pictures/P1/Sensors/voltage_sensors_placement.PNG}
		\caption{Voltage sensors placement.}
		\label{fig:voltage_sensors_placement}
	\end{center}
\end{figure} 
\fi

\paragraph{Voltage divider}
The input voltage at the voltage sensor is recommended to be in the range of $0V-2V$. To divide the measured voltage into that range, a voltage divider will be implemented. 

The STC output voltage of the PV module is the open-circuit voltage at $45V$. Since it is possible to have a higher $V_{OC}$ when the temperature is low, $50V$ has been selected as maximum voltage for $2V$ at the output of the voltage divider. The current flow in the voltage divider has been set at $1mA$, to secure a insignificant power loss. The resistors can be calculated with equation \ref{voltage_divider_R17_in} and \ref{voltage_divider_R18_in}.

\begin{equation} \label{voltage_divider_R17_in}
	R_{17} = \frac{V_{in,max}-V_{out}}{I} = \frac{50V-2V}{1mA} = 48k\Omega
\end{equation}

\begin{equation} \label{voltage_divider_R18_in}
	V_{out} = V_{in,max} \cdot \frac{R_{18}}{R_{17}+R_{18}} \Rightarrow 2V = 50V \cdot \frac{R_{18}}{48k\Omega+R_{18}}
\end{equation}
\begin{center}
	$R_{18} = 1.958k\Omega$
\end{center}

To be able to select commercial resistors the values $R_{17} = 47k\Omega$ and $R_{18} = 2k\Omega$ have been chosen. 

\paragraph{Filtering} \label{voltage_sensor_filter}
In order to reduce the noise and have a stable signal, a low-pass RC filter with a corner frequency at $50Hz$ will be placed between the voltage divider and the sensor. The corner frequency has been calculated by a thousandth of the switching frequency, which is $50 kHz$. The resistor of the filter will be $R_{17}$ in the voltage divider. The capacitor will be calculated as followed in equation \ref{voltage_sensor_in_filter_cap}:
\begin{equation} \label{voltage_sensor_in_filter_cap}
	C_{17} = \frac{1}{2\pi \cdot f_c \cdot R_{17}} = \frac{1}{2 \pi \cdot 50Hz \cdot 47k\Omega} = 67.7nF
\end{equation}

\noindent To be able to select a commercial capacitor, $C_{17} = 68nF$ has been chosen. 

\paragraph{Amplification} \label{voltage_sensor_amplification}
The input range of the Analog to Digital Converter (ADC) in the RT Box is $0V-5V$. To take advantage of the full range an amplifier will be implemented. The output of the voltage sensor is differential with an offset at $1.23V$. Therefore a differential amplifier will be implemented using a LMC6484 \cite{sensor_opamp}  quad operational amplifier. By using a quad amplifier, the same IC can be used for all the sensors. The relevant electrical specifications have been included in table \ref{tab:amplifier_specs}.


\begin{table}[H]
	\centering
	\begin{tabular}{|p{6cm}|>{\centering}p{3.5cm}|>{\centering}p{3.5cm}|}
		\hline
		\rowcolor{lightgray}\multicolumn{3}{|l|}{ \textbf{Recommended ratings}} \\ \hline
		Supply voltage 	& $V_{DD}$ 		& $3-15.5$ [V]  \tabularnewline \hline
		Input voltage range & $V_{in}$ 	& $\pm V_{DD}$ [V]  \tabularnewline \hline
		
		\rowcolor{lightgray}\multicolumn{3}{|l|}{ \textbf{Other values of interest}} \\ \hline
		Number of amplifiers 		&  	--	& $4$ [-]			\tabularnewline \hline
		Package 					&  	--	& SOIC [-] 				\tabularnewline \hline
		
	\end{tabular}
	\caption{Electrical specifications LMC6484 \cite{sensor_opamp}.}
	\label{tab:amplifier_specs}
\end{table}

\noindent The resistors of the differential amplifier will be sized with equation \ref{voltage_sensor_gain}.
\begin{equation} \label{voltage_sensor_gain}
	V_{out} = \frac{R_{21}}{R_{19}} \cdot (V_2-V_1)
\end{equation}

Where $V_2-V_1$ is the difference between the output pins of the voltage sensor. With unity gain in the voltage sensor, the maximum difference at the output will be $2V$. This should correspond to the maximum input voltage of the ADC at $5V$. $R_{19}$ is selected to be $11k\Omega$. The resistor $R_{21}$ is now calculated using equation \ref{voltage_sensor_gain}.
\begin{equation}
	5V = \frac{R_{21}}{11k\Omega} \cdot 2V
\end{equation}
\begin{center}
	$R_{21} = 27.5k\Omega$
\end{center}
To be able to select commercial resistors the value of $R_{21}$ it's rounded to be $27k\Omega$. Furthermore $R_{20} = R_{19}$ and $R_{22} = R_{21}$, to get a balanced differential amplifier.

\subsubsection{Output voltage sensor}
The voltage sensor at the output is designed with the same procedure as the input sensor. The maximum output voltage of the voltage divider is set to $2V$.

\paragraph{The circuit}
The circuit regarding the output voltage measurement is shown at figure \ref{fig:output_voltage_sensor_circuit}. 

\begin{figure}[H]
	\begin{center}
		\includegraphics[width=0.8\linewidth]{../Pictures/P1/Sensors/output_voltage_sensor.PNG}
		\caption{Output voltage sensor.}
		\label{fig:output_voltage_sensor_circuit}
	\end{center}
\end{figure}

\paragraph{Voltage divider}
The maximum output voltage of the DC-DC converter will be $90V$, when only 4 PV modules are used. To insert a safety margin if one converter fails, the maximum sensed voltage will be designed at $120V$.

\noindent The resistors will be sized by reusing equations \ref{voltage_divider_R17_in} and \ref{voltage_divider_R18_in}.

\begin{equation}
	R_{26} = \frac{V_{in,max}-V_{out}}{I} = \frac{120V-2V}{1mA} = 118k\Omega	
\end{equation}

\begin{equation} 
	V_{out} = V_{in,max} \cdot \frac{R_{27}}{R_{26}+R_{27}} \Rightarrow 2V = 120V \cdot \frac{R_{27}}{118k\Omega+R_{27}}
\end{equation}
\begin{center}
	$R_{27} = 2.03k\Omega$
\end{center}

\noindent To be able to select commercial resistors $R_{26} = 120k\Omega$ and $R_{27} = 2k\Omega$ have been chosen. 

\paragraph{Filtering}
The filter will be designed with the same corner frequency at $50Hz$, as for the input sensor.

The resistor of the filter will be $R_{26}$ in the voltage divider. The capacitor will be calculated as followed in equation \ref{voltage_sensor_out_filter_cap}:
\begin{equation} \label{voltage_sensor_out_filter_cap}
	C_{22} = \frac{1}{2\pi \cdot f_c \cdot R_{26}} = \frac{1}{2 \pi \cdot 50Hz \cdot 120k\Omega} = 26.5nF
\end{equation}

To be able to select a commercial capacitor $C_{22} = 33nF$ has been chosen. By using these values the actual corner frequency will be $40.2 Hz$.

