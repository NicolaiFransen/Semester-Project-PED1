\chapter{Conclusion}\label{ch:conclusion}


%https://www.wikihow.com/Write-a-Good-Lab-Conclusion-in-Science

%https://writingcenter.unc.edu/tips-and-tools/conclusions/

%https://www.wikihow.com/Write-a-Conclusion-for-a-Research-Paper


%\subsubsection{Restate topic}
During the last decades, pollution and climate change have become some of the most important issues that human kind is facing. The generation of renewable electric energy will play a major roll in the mitigation of these problems.

%\subsubsection{Restate statement}
One of the most promising technologies that will allow this transition is the photovoltaic generation. 
In order to maximize the efficiency of solar panels, a module integrated converter was designed.
The hardware was created and a maximum power point tracking algorithm was developed.

%\subsubsection{Main}
After coming up with the most adequate power topology, the system was divided into three main sections, hardware, control and software. All of which were developed, implemented and tested.
The creation of the hardware consisted on the sizing and selection of all the components, the design of a printed circuit board and the assembly. The control consisted of the research of existing methods and the development of a suitable algorithm. Finally the creation of a software that implements such algorithm was performed.

%\subsubsection{Improvements}
One of the biggest obstacles during the development of the project was the lack of feedback and protections of the hardware during the tests. This problem led to a slowdown on the testing process since the sources of the errors took a long time to be discovered. 
An implementation of error recognition hardware and software would have solved this problem, saving time and money.
%The work performed was designed to admit future improvements of the system. 
The main reason for making the topology bidirectional was allowing rapid crack recognition on the PVs, however this topic was not scoped during the project and thus there is room for future research.

%\subsubsection{So what}
On the road to a more environmentally friendly generation of electrical energy, the scope of this report is fundamental. Improving the efficiency of energy production entails more cost effective products, which will ultimately lead to the viability of renewable energy production worldwide.