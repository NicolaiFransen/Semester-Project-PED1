\section{PCB creation}
\subsection{Considerations} \label{PCB_Considerations}
Before starting the design of the PCB, some features were considered. First and most importantly, the power side, where all the big currents and voltages are going to be located is physically separated from the control side in order to reduce electromagnetic coupling. \todo{Long sentence consider rephrasing, NHF} A clear differentiation of the hot and cold sides of the PCB is seen in the layout.  Also, test points have been located in the periphery of the PCB for easiest and safest testing. Current loops have been undersized\todo{minimized?, NHF} as much as possible and ground planes are located across the PCB. \todo{Include image of the PCB layout. AT}

\subsection{Power Side} \label{PCB_Power}
One of the most important issues to take into consideration when designing the power traces of the converter has been the size of these. In order to allow a high flow of current through the traces while maintaining a relatively low temperature as well as a low impedance, the traces have been designed for minimum 10mm wide, with this thickness it has been calculated to have a temperature increase of 5ºC and an impedance of ¿?¿?¿?. \todo{include Nico's calculations. AT}  

The heat sink has conditioned the design of the power side of the PCB since most of the passive components do not fit underneath. The four MOSFETS are located at even distances and at the corners of the heat sink to allow the best possible dissipation. The drivers are also located very close to each one of the transistors to have the shortest path to the gate.

The coil has been cornered since it might induce high interference to other sensible devices, especially the current sensor, which is hall-effect has been located as far as possible from its influence. Also the ground plane has been removed from under the inductor in order to reduce interferences.

Finally, the high frequency capacitors are located very close to both power stages and the current loop area has been minimize to reduce the inductive behavior of this loop. 

\subsection{Control Side} \label{PCB_Control}
The control side has two different parts which are separated according to the ground that they have. On one side are located all the components that share the ground with the power side. This is the power supplies and the voltage dividers. The components that only have the low voltage ground, are located farer\todo{Is that a word? maybe further, NHF} away from the power circuit and finally some components act as bridges since they have both grounds. These components are the optocouplers and the voltage sensors, both with optical isolation.
 
This way, it is possible to separate the grounds at both sides reducing the possibility to having a short-circuit which may damage the control unit, in this case, the RT-box. However, the current sensor had to be located directly in the power side of the converter since the main current needs to flow through it. This means that the isolation would be broken. In order to reduce this problem as much as possible, higher clearances were included across the traces until they get to the low side.\todo{These sentences are pretty unclear, maybe try to rephrase them (from This means that...), NHF}

Since this is a testing version of the PCB, test points were located in the periphery of the PCB, breaking at some points this isolation clearance that was implemented in most parts of the circuit. Since test points would always be removed in a final version of the PCB, this problem has not been considered.\todo{This has been stated in the beginning of the section, and should be deleted, NHF}
 \todo{include pictures. AT}
