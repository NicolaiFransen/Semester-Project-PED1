\section{PCB design}
\subsection{PCB structure} \label{PCB_Schematic}
In order to proceed with the creation of the PCB, a schematic circuit has been designed. This compiles all the previously mentioned components as well as other components required for current limiting, decoupling, external connections, or safety components for protection. Both PCB and schematic might be found in appendix \ref{ch:AppPCB}.

The schematic has been divided into four main sections, these are \textit{main topology}, \textit{power supplies}, \textit{drivers} and \textit{signal processing}.

The main topology includes the power circuit, this is the MOSFETs, the inductor, the input and output capacitors and the input and output connectors. It also includes discharging resistors for the capacitors which have been sized for one minute discharge time. The gate of the MOSFETs is also connected to the source through a resistor designed for 5ms discharge time. These safety resistors are implemented in order to ensure that the circuit is fully discharged when disconnected. Finally a series Schottky diode is included at the input to protect the components in case of reverse connection. However, this diode can be short-circuited at any moment.

The power supplies include the 12V external input connector and two isolated commutative 12V. These two supplies are necessary to power the gates of the high side transistors and drivers. Lastly, a 5V linear regulator supply has also been connected to feed the sensors. This 5V signal is referred to the power ground, thus the input 5V signal could not be used since it does not share the same ground.

The drivers section is composed of the isolating optocouplers and the MOSFET gate drivers. The PWM signals to the optocouplers come directly from the RT-box, meaning that a connector is also included here. Finally decoupling capacitors are added in the vicinity of the ICs in order to assure that the current peaks needed by these are granted.

The signal processing section is composed by the IC sensors and the operational amplifier. As stated previously, there are 2 isolated voltage sensors and a hall effect sensor for current measuring. Also, voltage dividers are located in this section, these include protection zener diodes limiting 5V output voltage and filtering capacitors, finally amplification resistors are added. The current sensor also includes a selection pin header which will allow different filtering frequencies options. The operational amplifier is also included.

Finally, test points have been added to the signals that might be measured.
