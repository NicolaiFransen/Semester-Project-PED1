%%% Chapter 5 - Hardware Implementation %%%

\chapter{Hardware implementation} \label{ch:hardware_implementation}
%%% Introduction to the chapter: Hardware implementation %%%

This chapter contains the hardware implementation of the non-inverting buck-boost converter. This mainly consists of the selection of commercial components, including switches, drivers, sensors, passive components etc. Additional circuits will also be designed to support the selected components. For the switch selection, power loss and temperature dissipation will be considered. This chapter also includes creation of the schematics and the PCB layout for the entire hardware implementation. 

\section{Selection of commercial components}

\subsection{Passive components}
\todo{WE MADE A MISTAKE HERE, NEED TO EXPLAIN WHY WE OBTAINED A MUCH LOWER VALUE. MAYBE TO EMULATE THE CAPACITANCE AT THE INVERTER.}
For the output capacitor a $820\mu F$\todo{in equation 4.9 we have a completely different value.. why? Stef } has been used \cite{cout}. An electrolytic capacitor has mostly been chosen because of small size and low cost. The voltage rating is 250V which is a fine margin to the 90V that is the highest possible output. \todo{with max 90V a much lower voltage rating could have been chosen! margin is again not the reason for this value, true that maybe we did oversize it when designing. AT} \todo{well not saying that the margin is the reason. Just mentioning it. Can't see that is a problem? JK}
Achieving the necessary input capacitance with only one capacitor would result in very big size of the component, therefore four electrolytic capacitors of $470\mu F$ \cite{cin} are placed in parallel to get a summed capacitance of $1.880\mu F$ \todo{1.8 mF isn't it? Stef}. The rating for these is 100V which has been chosen in case the output is connected to the input by accident. In this case that will not ruin the input capacitor. These ESR resistances\todo{which ESR resistance? and how do you get this value?Stef} will be about $0.047\Omega$. Both input and output capacitor have a capacitance higher than the calculated worst case. This is because the converter maybe in the future will have a buck-boost mode as well. And in this mode the capacitance can get a bit higher. \todo{I think this is not the best way to introduce the buck boost possibil<ity. AT}
Both for the input and output a $100nF$ and a $1\mu F$ \todo{where do these values come from?Stef} capacitors are placed in parallel. This is mostly done because the big electrolytic capacitors is not able to supply the needed current at higher frequencies. With these extra capacitors the high frequency transients should be filtered away.

The inductor for first iteration \todo{What does first iteration mean? have we explained it previously? AT} will be with an inductance of around 1mH, \todo{exact value should be determined JK} because that is an inductor the supervisors had in-house already.

\subsection{Swithing circuitry}
Selecting the switching circuitry consists of four different parts, the switches, drivers, optocouplers and the heat sink. Additional support circuitry will be designed, when selecting the commercial components.

\subsubsection{Switch sizing} \label{switch_sizing}
The system must regulate the power flow in order to maximize the power generation. In order to achieve this, the system includes switches that control the current flow. The switches consist on MOSFET devices. The switching frequency of the system is selected to be $50kHz$. Although the market has IGBT which can switch at $50kHz$, MOSFET devices allow lower losses than IGBTs for system's current rating \cite{mosfet_igbt_switching_loss} \cite{igbt_or_mosfet}.


The maximum output voltage of the system is $90V$, however the voltage rating of the transistors was set to $150V$ in order to consider a safety margin, and thus, increase the reliability. The peak current through the transistors happens when the buck mode is active and the maximum number of MICs are used in series. The peak current is equal to $14A$. In order to reduce the conduction losses and the heat sink size, a low on resistance is desired shown in equation \ref{conduction_losses_eq}\cite{mosfet_losses}. These constraints were used when searching for the ideal component. The chosen device is the IPB200N15N3. It exhibits the features seen in table \ref{mosfet_features}.


\begin{table}[htbp]
	\centering
	\begin{tabular}{|p{6cm}|>{\centering}p{6cm}|}
		\hline
		\rowcolor{lightgray}\multicolumn{2}{|l|}{ \textbf{Maximum ratings}} \\ \hline
		Continuous $I_{D}$ & 40 [A]  \tabularnewline \hline
		$V_{GS}$ & $\pm$ 20 [V]  \tabularnewline \hline
		Power dissipation & 150 [W]  \tabularnewline \hline
		$V_{DS}$ & 150 [V]  \tabularnewline \hline
		$R_{DSon} $ & 20 [m$\Omega$]  \tabularnewline \hline
		\rowcolor{lightgray}\multicolumn{2}{|l|}{ \textbf{Other values of interest}} \\ \hline
		Input capacitance & 1820 [pF]  \tabularnewline \hline
		Package & D2PAK  \tabularnewline \hline
		$V_{th} $ $(V_{GS} = 3 V)$ & 3 [V]  \tabularnewline \hline
		$V_{th} $ $(V_{GS} = 36 V)$ & 4.7 [V]  \tabularnewline \hline
		$R_{Gate} $ & 2.4 [$\Omega$]  \tabularnewline \hline
	
	\end{tabular}
	\caption{MOSFET figures of merit. T = 25 $\decC$ \cite{mosfet_datasheet}.}
	\label{mosfet_features}
\end{table}

\subsubsection{Heat sink sizing}

The procedure followed for validating the heat sink might be seen at figure \ref{heat_sink_validation_procedure}. If the total temperature increase is within switch's safe operating area, then the heat sink is providing enough heat dissipation.

\begin{figure}[htbp]
	\begin{center}
		\includegraphics[width=\textwidth]{../Pictures/P1/Component_sizing/heat_sink_validation_procedure.png}
		\caption{Heat sink validation procedure.}
		\label{heat_sink_validation_procedure}
	\end{center}	
\end{figure}

The power dissipated in the switches is equal to the sum of the conduction losses and the switching losses. The conduction losses might be calculated  as seen in equation \ref{conduction_losses_eq}.

\begin{equation} \label{conduction_losses_eq}
P_{cond} = i(t)^2 \cdot R_{DS}
\end{equation}

The switching losses depend upon the switching frequency and the transistor's manufacturing characteristics\cite{mosfet_losses}. In order to calculate the value, the MOSFET's SPICE model was obtained from the manufacturer's website. The next step was to perform the simulation of the system. The system was simulated in both buck and boost modes. Special attention was put into the dead-band between PWM signals of different switches, to avoid current shoot through. After simulating, the average power dissipation under steady state was calculated in both modes. Within every mode, the simulation was performed under the most unfavorable conditions, this is: buck's output is 24 V and boost's output is 90 V. The results can be seen in table \ref{mosfet_final_dissipation}, column 1.

The datasheet shows that the $R_{DS}$ variation is mainly dependent on the temperature\cite{mosfet_datasheet}. The models found neglect the temperature difference. Then, in order to get an approximated value considering temperature, the procedure will be to calculate the total losses at constant temperature using the SPICE model and then add the additional conduction losses due to the increase of the resistance, as expressed by equation \ref{total_losses}.

\begin{equation} \label{total_losses}
\overline{P} = \overline{P_{loss, T = K}} + \overline{i(t)^2 \cdot \Updelta R_{DS}}
\end{equation}

Now the junction temperature based on the power dissipation, calculated using the SPICE model, is calculated. The ambient temperature is set to 50 $\decC$, which is considered a realistic scenario. The thermal circuit can be seen in figure \ref{thermal_circuit}. The next step is to choose a commercial heat sink. The constraints are thermal resistance, size and price. TDEX6015/TH was found. Its features might be found in table \ref{heatsink_features}. The switches temperature will be analysed in order to validate the heat sink. The analysis considers all the transistors as a single power source.

\begin{figure}[H]
	\begin{center}
		\includegraphics[width=0.7\textwidth]{../Pictures/thermal_circuit.png}
		\caption{Thermal circuit used for sizing the heat sink}
		\label{thermal_circuit}
	\end{center}	
\end{figure}

\begin{equation} \label{switch_temperature}
T_{J} = T_{housing} + \overline{P_{loss, T = K}} \cdot  R_{thermal}
\end{equation}


If no heat sink were used, according to equation \ref{switch_temperature}, the junction temperature would become too high and the components would be damaged. See equation \ref{temperature_without_heatsink}. This is mainly explained due to the fact that the thermal resistance between junction and ambient of the transistor is as high as 75 $\decC / W$.

\begin{equation} \label{temperature_without_heatsink}
T_{J} = 50 \decC + 5.54 W \cdot 75 \frac{\decC}{W} = 465.5 \decC
\end{equation}


\begin{table}[htbp]
	\centering
	\begin{tabular}{|p{6cm}|>{\centering}p{8cm}|}
		\hline
		\rowcolor{lightgray}\multicolumn{2}{|l|}{ \textbf{Features}} \\ \hline
		Size & 60x60x16 [mm]  \tabularnewline \hline
		Thermal resistance & 2.06 [K/W]  \tabularnewline \hline
		
	\end{tabular}
	\caption{Heat sink figures of merit \cite{heatsink_datasheet}.}
	\label{heatsink_features}
\end{table}


\begin{equation} \label{switch_temperature_w_values}
T_{J} = 50 \decC + 5.54 W \cdot  2.06\frac{\decC}{W} = 61.41 \decC
\end{equation}

The drain-to-source resistance increase is calculated as explained in equation \ref{delta_resistance}. The resistance difference is relatively small. The resistor at every temperature was collected from the component data sheet.

\begin{equation} \label{delta_resistance}
\Updelta R_{DS} = |R_{DS, T = 20 \decC} - R_{DS, T = 61.41\decC}| = 4\; m \Omega
\end{equation}

\begin{table}[]
	\centering
	\begin{tabular}{|l|l|l|l|}
		\hline
		\rowcolor[HTML]{C0C0C0} 
		\multicolumn{4}{|c|}{\cellcolor[HTML]{C0C0C0}\textbf{Switches power dissipation}}                                                   \\ \hline
		\rowcolor[HTML]{C0C0C0} 
		Switch         & $\overline{P_{loss, T = K}}$ {[}W{]} & $ \overline{i(t)^2 \cdot \Updelta R_{DS}}$ {[}W{]} & \textbf{Total {[}W{]}} \\ \hline
		\multicolumn{4}{|l|}{Buck mode}                                                                                                     \\ \hline
		M1             & 2.91                                 & 0.39                                               & \textbf{3.30}          \\ \hline
		M2             & 0.82                                 & 0.21                                               & \textbf{1.03}          \\ \hline
		M3             & 1.81                                 & 0.58                                               & \textbf{2.39}          \\ \hline
		M4             & 0                                    & 0                                                  & \textbf{0}             \\ \hline
		\textbf{Total} & 5.54                                 & 1.18                                               & \textbf{6.72}          \\ \hline
		\multicolumn{4}{|l|}{Boost mode}                                                                                                    \\ \hline
		M1             & 0.69                                 & 0.28                                               & \textbf{0.97}          \\ \hline
		M2             & 0                                    & 0                                                  & \textbf{0}             \\ \hline
		M3             & 0.48                                 & 0.12                                               & \textbf{0.6}           \\ \hline
		M4             & 3.31                                 & 0.18                                               & \textbf{3.49}          \\ \hline
		\textbf{Total} & 4.48                                 & 0.58                                               & \textbf{5.06}          \\ \hline
	\end{tabular}
\caption{Power dissipation analysis. Column 1, average power dissipation at constant 25 $\decC$ temperature. Column 2, extra power dissipation due to the increase of temperature.}
\label{mosfet_final_dissipation}
\end{table}


The full power dissipation values can be found on table \ref{mosfet_final_dissipation}. To achieve an exact result, an iterative process should be followed. However, after the first iteration, the change ratio is extremely small and then, neglected. Now that the power dissipation has been calculated, the junction temperature must be checked in order to confirm that the heat sink has been properly sized. Equation \ref{switch_temperature} is used, substitution of values leads to \ref{switch_temperature_w_values_2}. The difference is fairly small and the junction temperature remains within safe area. Then, TDEX6015/TH has been validated as a proper heat sink.

\begin{equation} \label{switch_temperature_w_values_2}
T_{J} = 50 \decC + 6.72 W \cdot  2.06 \frac{\decC}{W} = 63.84 \decC
\end{equation}
\subsection{Driver}

The converter was not working properly in boost mode. However, it was possible to get some test results from it before failure. 
When this happened the driver of MOSFET 4 broke and needed to be changed. The reason is still being analyzed. 
Voltage spikes were found at the gate of the MOSFET which may be leading to failure. Figure \ref{Voltagespike} shows the output voltage of the driver after the gate resistors R11 and R12:

\begin{figure}[H]
	\begin{center}
		\includegraphics[width=0.7\textwidth]{../Pictures/P1/Discussion/Voltagespike.jpg}
		\caption{Voltage at the gate of MOSFET 4}
		\label{Voltagespike}
	\end{center}	
\end{figure}    

Looking at the figure, it can be seen that the switching spikes are reaching above 50V. This means that a large current will flow through the driver and MOSFET while switching which might be the cause of the driver failures. In an attempt to lowering these voltage spikes, the gate resistors were changed from $20\Omega$ to $100\Omega$. Improvements were obtained and are shown in figure \ref{Voltagespike_100}.
The drawback of increasing the gate resistors is an increase of the response time at the gate of the MOSFET. This behavior can also be seen in the figure.

\begin{figure}[H]
	\begin{center}
		\includegraphics[width=0.7\textwidth]{../Pictures/P1/Discussion/Voltagespike_new_resistors.jpg}
		\caption{Output voltage of driver after gate resistors $100\Omega$}
		\label{Voltagespike_100}
	\end{center}	
\end{figure} 

As expected, the changed resistors have decreased the voltage spikes significantly. However, this did not solve the problem since the driver were still breaking. 

%% These sections describes the design and implementation of the chosen sensors.

\section{Sensors} \label{sensors}
To implement the MPPT it's necessary to measure the output voltage and current of the PV-module. These measurements will be obtained by implementing a voltage and current sensor. A second voltage sensor will be implemented to measure the output voltage of the DC/DC coverter, for possible future use. 

To protect the RT-Box, it has been chosen to fully isolate it from the power stage of the converter. To do so, the sensors will have to include isolation between input and output.  

%%% Voltage sensors %%%
\subsection{Input voltage sensor} \label{voltage_sensors}
The voltage sensor selected is ACPL-C870 \cite{voltage_sensor}. This sensor includes optical isolation amplifiers, which makes it well suited for isolated voltage sensing. The amplifier includes unity gain $1V/V$ amplification, with an accuracy at $\pm 3 \%$. 

\subsubsection{Voltage divider}
The input voltage at the voltage sensor is recommended to be in the range of $0V-2V$. To divide the measured voltage into that range, a voltage divider will be implemented. 

The maximum output voltage of the PV-module is the open-circuit voltage at $46V$\todo{Get the correct value}. To achieve a safety margin and to increase integration with other types of PV-modules\todo{Write a better description of this}, $50V$ has been selected. The current flow in the voltage divider has been set at $1mA$, to secure a insignificant power loss. The resistors can be calculated with the following equations:
\begin{equation} \label{voltage_divider_R1_in}
	R_1 = \frac{V_{in,max}-V_{out}}{I} = \frac{50V-2V}{1mA} = 48k\Omega
\end{equation}

\begin{equation} \label{voltage_divider_R2_in}
	V_{out} = V_{in,max} \cdot \frac{R_2}{R_1+R_2} \Rightarrow 2V = 50V \cdot \frac{R2}{48k\Omega+R2}
\end{equation}
\begin{center}
	$R_2 = 1.958k\Omega$
\end{center}

To achieve these resistor values $R_1 = 47k\Omega$ and $R_2 = 2k\Omega$ have been chosen. 

\subsubsection{Filtering} \label{voltage_sensor_filter}
For a stable MPPT control, the measured voltage must have a very low ripple. To ensure this, a low-pass RC filter with a corner frequency at $50Hz$, will be placed between the voltage divider and the sensor. The resistor of the filter will be $R_1$ in the voltage divider. The capacitor will calculated as followed in equation \ref{voltage_sensor_in_filter_cap}:
\begin{equation} \label{voltage_sensor_in_filter_cap}
	C_1 = \frac{1}{2\pi \cdot f_c \cdot R_1} = \frac{1}{2 \pi \cdot 50Hz \cdot 47k\Omega} = 67.7nF
\end{equation}\todo{Check for available values}

\subsubsection{Amplification} \label{voltage_sensor_amplification}
The input range of the ADC in the RT-Box is $0V-5V$. To take advantage of the entire range an amplifier will be implemented. The output of the voltage sensor is differential with an offset at $1.23V$. Therefore a differential amplifier will be implemented using a LMC6484 quad operational amplifier \cite{sensor_opamp}. By using a quad amplifier, the same IC can be used for the output voltage sensor and the current sensor.

The resistors of the differential amplifier will be sized with equation \ref{voltage_sensor_gain}.
\begin{equation} \label{voltage_sensor_gain}
	V_{out} = \frac{R_3}{R_1} \cdot (V_2-V_1)
\end{equation}

Where $V_2-V_1$ is the difference between the output pins of the voltage sensor. With unity gain in the voltage sensor, the maximum difference at the output will be $2V$. This should correspond to the maximum input voltage of the ADC at $5V$. $R_1$ is selected to be $11k\Omega$. The resistor $R_3$ is now calculated using equation \ref{voltage_sensor_gain}.
\begin{equation}
	5V = \frac{R_3}{11k\Omega} \cdot 2V
\end{equation}
\begin{center}
	$R_3 = 27.5k\Omega$
\end{center}
To achieve the value of $R_3$ it's rounded to be $27k\Omega$. 

\subsection{Output voltage sensor}
The voltage sensor at the output is design by the same procedure as the input sensor. The voltage divider will designed such that the values of the amplifier can be reused.

\subsubsection{Voltage divider}
The maximum output voltage of the DC/DC converter will be $90V$, when only 4 PV-modules are used. To insert a safety margin if one converter fails, the maximum sensed voltage will be designed at $120V$.

The resistors will be sized by reusing equation \ref{voltage_divider_R1_in} and \ref{voltage_divider_R2_in}.
\begin{equation}
	R_1 = \frac{V_{in,max}-V_{out}}{I} = \frac{120V-2V}{1mA} = 118k\Omega	
\end{equation}

\begin{equation} 
V_{out} = V_{in,max} \cdot \frac{R_2}{R_1+R_2} \Rightarrow 2V = 120V \cdot \frac{R2}{118k\Omega+R2}
\end{equation}
\begin{center}
	$R_2 = 2.03k\Omega$
\end{center}

To achieve these resistor values $R_1 = 120k\Omega$ and $R_2 = 2k\Omega$ have been chosen. 

\subsubsection{Filtering}
The filter will be design with the same corner frequency at $50Hz$, as for the input sensor.

The resistor of the filter will again be $R_1$ in the voltage divider. The capacitor will calculated as followed in equation \ref{voltage_sensor_out_filter_cap}:
\begin{equation} \label{voltage_sensor_out_filter_cap}
C_1 = \frac{1}{2\pi \cdot f_c \cdot R_1} = \frac{1}{2 \pi \cdot 50Hz \cdot 120k\Omega} = 26.5nF
\end{equation}\todo{Check for available values}

%%% Current sensors %%%
\subsubsection{Current sensor} \label{current_sensor}

The current along with the voltage of the PV allows the system to perform power calculation, which is needed for the MPPT algorithm. The current will be measured in parallel with the inductor with a hall effect sensor. Placing it in series with the PV module would be the easiest approach for MPPT, but placing it in parallel with the inductor allows implementing a current controller for possible future use.

\begin{figure}[htbp]
	\begin{center}
		\includegraphics[width=0.6\textwidth]{../Pictures/current_sensor_placement.png}
		\caption{Current sensor placement.}
		\label{current_sensor_placement}
	\end{center}	
\end{figure}

The sensor is a ACS723-20AB \cite{current_sensor} which is a Hall effect sensor. Its features might be found in table \ref{current_sensor_features} and its connection might be found in \ref{current_sensor_application}.

\begin{table}[htbp]
	\centering
	\begin{tabular}{|p{6cm}|>{\centering}p{8cm}|}
		\hline
		\rowcolor{lightgray}\multicolumn{2}{|l|}{ \textbf{Maximum ratings}} \\ \hline
		Supply voltage & 4.5-5.5 [V]  \tabularnewline \hline
		Gain & 100 [mV/A]  \tabularnewline \hline
		Input range & $\pm$20 [A]  \tabularnewline \hline
		\rowcolor{lightgray}\multicolumn{2}{|l|}{ \textbf{Other values of interest}} \\ \hline
		Bandwidth & 20 or 80 [kHz]  \tabularnewline \hline
		Package & SOIC8  \tabularnewline \hline
		
	\end{tabular}
	\caption{Current sensor figures of merit. \cite{current_sensor}}
	\label{current_sensor_features}
\end{table}

\begin{figure}[htbp]
	\begin{center}
		\includegraphics[width=\textwidth]{../Pictures/P1/Sensors/current_sensor}
		\caption{Current sensor connection.}
		\label{current_sensor_application}
	\end{center}	
\end{figure}

The output of the sensor is a voltage proportional to the current following the next equation:

\begin{equation} 
V_{current} = \frac{1}{10} \, i + 2.5
\end{equation}

In order to ease the task of the control, the signals are filtered by hardware. The current will be used by the MPPT, which frequency is $100 Hz$ \todo{check final implementation}. The sensor output is filtered by a LPF which cut-off frequency is $50 Hz$. The cut-off frequency has been calculated by a thousandth of the switching frequency, which is $50 kHz$. Also the current might be used in the current controller, this signal will be filtered at $80 KHz$ in order to remove high frequency noise, this cut-off frequency was selected as it is the sensor's bandwidth. The filters are first order low-pass filters implemented with a resistor in series with a capacitor.\todo{has the current control been implemented? was enough this filtering?}

In order to calculate the current from the PV module, the converter working mode will have to be taken into account. Assuming continuous conduction mode, the average PV current is:


\begin{equation} 
	Buck \; mode \rightarrow \overline{I_{in}} = \overline{i_{measured}} \cdot \delta
\end{equation}
\begin{equation} 
Boost \; mode \rightarrow \overline{I_{in}} = \overline{i_{measured}}
\end{equation}
\begin{equation} 
Buck-Boost \; mode \rightarrow \overline{I_{in}} = \overline{i_{measured}} \cdot \delta
\end{equation} 


\section{Power Supplies}\label{power_supplies}
In the first iteration of the converter, the drivers and the sensors will be supplied by an external $12V$ voltage source. This source will be used directly to supply the two lower leg MOSFET drivers. To support the higher leg drivers two isolating $12V$ supplies will be used, with the external $12V$ as input. These will be two TRACO supply \textit{TMA1212S}. The chosen voltage sensors need a $5V$ power supply at both input and output of the sensor. These should isolated from each other. The input side will be supplied by a $5V$ voltage regulator, \textit{LD1117}, and the output side will be supplied by the RT-box. The current sensor will also be supplied with $5V$ by the RT-box.


\section{PCB design}
\subsection{PCB structure} \label{PCB_Schematic}
In order to proceed with the creation of the PCB, a schematic circuit has been designed. This compiles all the previously mentioned components as well as other components required for current limiting, decoupling, external connections, or safety components for protection. Both the PCB layout and schematics are found in appendix \ref{ch:AppPCB}.

The schematic has been divided into four main sections, these are \textit{main topology}, \textit{power supplies}, \textit{drivers} and \textit{signal processing}.

The main topology includes the power circuit, this is the MOSFETs, the inductor, the input and output capacitors and the input and output connectors. It also includes discharging resistors for the capacitors which have been sized for one minute discharge time. The gate of the MOSFETs is also connected to the source through a resistor designed for 3ms discharge time. These safety resistors are implemented in order to ensure that the circuit is fully discharged when disconnected. A series Schottky diode is included at the input to protect the components in case of reverse connection. However, this diode can be short-circuited at any moment.

The power supplies include the 12V external input connector and two isolated 12V supplies. These two are necessary to power the gates of the high side transistors and drivers. Lastly, a 5V linear regulator supply has also been connected to feed the sensors. This 5V signal is referred to the power ground, thus the input 5V signal could not be used since it does not share the same ground.

The drivers section is composed of the isolating optocouplers and the MOSFET gate drivers. The PWM signals to the optocouplers come directly from the RT Box, meaning that a connector is also included here. Decoupling capacitors are added in the vicinity of the ICs in order to assure that the current peaks needed by these are granted.

The signal processing section is composed by the IC sensors and the operational amplifier. As stated previously, there are 2 isolated voltage sensors and a hall effect sensor for current measuring. Also, voltage dividers are located in this section. These include protection zener diodes limiting 4.7V output voltage and filtering capacitors, amplification resistors are added. The current sensor also includes a selection pin header which will allow different filtering frequencies options. The operational amplifier is also included.

Finally, test points have been added to the signals that might be measured.


\input{docs/hardwareImplementation/PCB_layout/PCB_layout.tex}