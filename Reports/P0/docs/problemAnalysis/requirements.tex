\section{System requirements}

For the design and test of the MIC it is of great importance to have the requirements of the system defined. The group decided to develop a DC-DC converter to integrate it with a generic PV module which is capable to deliver a maximum power of 300 W. The electrical characteristics of the PV panel under Standard Test Conditions (STC) are shown in table \ref{el_charact_PV_panel}:

\begin{table}[H]
	\centering
	\begin{tabular}{ |l|c| } 
		\hline
		Maximum power (Pmax) & 300 [W]  \\ \hline
		Optimum Operating Voltage (Vmpp) & 32.6 [V]  \\ \hline
		Optimum Operating Current (Impp) & 9.21 [A]  \\ \hline
		Open Circuit Voltage (Voc) &  39.9 [V]\\ \hline
		Short Circuit Current (Isc) & 9.65 [A]  \\ \hline
		Module Efficiency ($\eta$) & 18.3 \%  \\ \hline
	\end{tabular}
	\caption{Electrical characteristics \textit{STP300S-20/Wfb} PV panel.[REF]}
	\label{el_charact_PV_panel}
\end{table}

The values from the previous table will be the input for the DC-DC converter. The group decided to connect the MIC's output to a commercial inverter \textit{"Power-one STGU-105"} in order to have the output voltage defined. From the inverter's datasheet it is found that the nominal voltage in the DC-link is 360 V. 
The development of this project will be based on these requirements because they are based on real commercial products that the user can purchase.


